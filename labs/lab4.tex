%!TEX root = ../main.tex
\chapter{Differential Equations}

\section*{Task 1}
In this question we compare the convergence of a few methods we have done in class.
\[ \dot{y}=\frac{y+t}{y-t}, \quad y(0)=1. \]
\begin{enumerate}[(a)]
	\item MATLAB\texttrademark\ can solve simple differential equations symbolically.
	For example,
	\begin{lstlisting}[style=Matlab-editor]
equation = 'Dy = (y+t)/(y-t)';
initial = 'y(0) = 1';
y = dsolve(equation, initial, 't');
pretty(y)
x = linspace( 0, 1, 20);
z = eval(vectorize(y));
plot(x,z)
	\end{lstlisting}
	Run the above commands and write down the exact solution below.
	\item Study and use the M-function {\color{blue}{butcher}} and \hyperref[euler]{euler} to estimate the convergence rate of Euler's method for the above IVP over the interval \([0, 1]\).
	In addition, compute the error as the difference between your approximation and the exact solution at \(t = 1\).
	\item Repeat part (b), but this time do it for Heun's method.
	The M-function \hyperref[heun]{heun} is given.
	\item Repeat to part (b), but this time do it for Aadms-Bashforth's method.
	The M-function \hyperref[ab2]{ab2} is given.
	\item Repeat to part (b), but this time do it for the classic 4-order Runge Kutta's method.
	The M-function \hyperref[rk4]{rk4} is given.
	\item Repeat to part (b), but this time do it for the classic 4-order Taylor's method.
	The M-function \hyperref[taylor]{taylor} is given.
\end{enumerate}


\section*{Task 2}
MATLAB\texttrademark\ has its implementation of Runge-Kutta.
This question looks at the build-in functions {\color{blue}ode23} and {\color{blue}ode45}, which implement versions of 2nd/3rd-order Runge-Kutta and 4th/5th-order Runge Kutta, respectively.
\begin{enumerate}[(a)]
	\item ) Study and use {\color{blue}ode23} to solve the following
	\[ \dot{y}=\frac{y+t}{y-t}, \quad y(0)=1. \]
	Compare and comment the convergence with the methods we considered in Task 1.
	\item Write an M-function for the vector-valued function
	\[ \Phi(t,x)=\begin{bmatrix} 10(x_2-x_1) \\ 28x_1-x_2-x_1x_3 \\ -8/3x_3+x_2x_2 \end{bmatrix} \]
	The input of your function shall be a scalar \(t\) and a vector in \(\R^3\).
	The output of your function shall be a column vector.
	\item  Study and use {\color{blue}ode45} to solve the following IVP over \([0, 20]\).
	\[ \dot{x}=\Phi(x), \quad x(0)=\begin{bmatrix} -8 \\ 8 \\ 27 \end{bmatrix}. \]
	\item Let \(X\) denote the matrix that contains approximations you have found in part (c) for \(x_1\), \(x_2\) and \(x_3\) as its columns at various tk values.
	Plot the following
	\begin{lstlisting}[style=Matlab-editor]
plot(X(:,1), X(:,2))
plot(X(:,1), X(:,3))
plot(X(:,2), X(:,3))
subplot(3,1,1)
plot(t,X(:,1))
subplot(3,1,2)
plot(t,X(:,2))
subplot(3,1,3)
plot(t,X(:,3))
	\end{lstlisting}
	What do those graphs suggest?
	What happens if we change the initial condition \(x_0\)?
\end{enumerate}


\section*{Task 3}
There are IVPs that defeat the explicit Adams-Bashforth and Runge-Kutta methods.
In fact, for such problems, the higher order method perform even more poorly than the low
order methods.
These problems are call ``stiff'' ODEs.
Consider
\[ y' = \lambda(-y+\sin x),\quad y(0) = 0. \]
whose exact solution is
\[ y(x)=\frac{\lambda}{1+\lambda^2}e^{-\lambda x}+\frac{\lambda^2}{1+\lambda^2}\sin x-\frac{\lambda}{1+\lambda^2}\cos x. \]
\begin{enumerate}[(a)]
	\item Run the following commands.
	The M-functions {\color{blue}{stiff2ode}} and {\color{blue}{stiff2solution}} are given.
	\begin{lstlisting}[style=Matlab-editor]
clear all
h = 0.1; % mesh size
[x,y] = meshgrid ( 0:h:2*pi, -1:h:1 );
px = ones ( size ( x ) );
py = stiff2ode ( x, y );
quiver ( x, y, px, py )
axis tight equal
hold on
x1=(0:h:2*pi);
y1=stiff2solution(x1);
plot(x1,y1,'r')
hold off
	\end{lstlisting}
	What does the direction field seem to suggest?
	What happens if \(\lambda\) increases?
	Describe your understanding of stiffness.
	\item Use \hyperref[euler]{euler} to solve the IVP when \(\lambda = 10000\) for 40 steps over the interval \([0, 2\pi]\).
	Comment your solution and provide an explanation of it.
	\item Use \hyperref[rk4]{rk4} to solve the IVP when \(\lambda = 10000\) for 40 steps over the interval \([0, 2\pi]\).
	Is it better than Euler's?
	What happens if we increase the number of steps?
	\item Study and use the build-in function ode15s to solve the IVP when \(\lambda = 10000\).
	Write down the estimated value of \(y(2\pi)\).
\end{enumerate}


\section*{Task 4}
Consider the following boundary value problem.
\[ \ddot{y}-3\dot{y}+2y=0, \quad y(0)=0, \quad y(1)=10. \]
\begin{enumerate}[(a)]
	\item Study and use the built-in function {\color{blue}{bvp4c}} to solve the above BVP over \([0, 1]\).
	Write the estimated value of \(y(1/\pi)\).
	\item Write an M-function to solve the above BVP using the shooting method.
	\item Write an M-function to solve the above BVP using the finite-difference.
\end{enumerate}
