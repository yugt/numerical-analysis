%!TEX root = ../main.tex
\chapter{Differential Equations}

\section{}
In this question we compare the convergence of a few methods we have done in class.
\[ \dot{y}=\frac{y+t}{y-t}, \quad y(0)=1. \]
\begin{enumerate}
	\item Matlab can solve simple differential equations symbolically.
	For example,
	\begin{lstlisting}[style=Matlab-editor]
equation = 'Dy = (y+t)/(y-t)';
initial = 'y(0) = 1';
y = dsolve(equation, initial, 't');
pretty(y)
x = linspace( 0, 1, 20);
z = eval(vectorize(y));
plot(x,z)
	\end{lstlisting}
	Run the above commands and write down the exact solution below.
	\item Study and use the M-function butcher and euler to estimate the convergence rate of Euler's method for the above IVP over the interval \([0, 1]\).
	In addition, compute the error as the difference between your approximation and the exact solution at \(t = 1\).
	\lstinputlisting[style=Matlab-editor]{mcode/butcher.m}
	\lstinputlisting[style=Matlab-editor]{mcode/euler.m}
	\item Repeat part (b), but this time do it for Heun's method.
	The M-function heun is given.
	\lstinputlisting[style=Matlab-editor]{mcode/heun.m}
	\item Repeat to part (b), but this time do it for Aadms-Bashforth's method.
	The M-function ab2 is given.
	\lstinputlisting[style=Matlab-editor]{mcode/ab2.m}
	\item Repeat to part (b), but this time do it for the classic 4-order Runge Kutta's method.
	The M-function rk4 is given.
	\lstinputlisting[style=Matlab-editor]{mcode/rk4.m}
	\item Repeat to part (b), but this time do it for the classic 4-order Taylor's method.
	The M-function taylor is given.
	\lstinputlisting[style=Matlab-editor]{mcode/taylor.m}
\end{enumerate}


\section{}
Matlab has its implementation of Runge-Kutta.
This question looks at the build-in functions ode23 and ode45, which implement versions of 2nd/3rd-order Runge-Kutta and 4th/5th-order Runge Kutta, respectively.