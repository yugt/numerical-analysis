%!TEX root = ../main.tex
\titleformat{\section}{\normalfont\Large\bfseries}{Task \thesection}{1em}{}
\chapter{Root Finding}

\section*{Task 1}
In this question, we will continue our study on how to write an M-function.
\begin{enumerate}[(a)]
	\item Study the M-function {\color{blue}{myplot}} and write down what the variable varargin in an M-function does.
	\begin{lstlisting}[style=Matlab-editor]
n = 20; xmin = -2; xmax = 2;
x = linspace( xmin, xmax, n);
y = x.^2 - 2;
myplot(x, y,'$$y=x^2-\ln(x^2+2)$$', 'color', 'b', 'linestyle', '-.')
	\end{lstlisting}
	\item  Write down what the variables {\color{blue} nargin}, {\color{blue} argout} and {\color{blue} varargout} in an M-function do.
	\item  Study and write down the output of the following commands.
	\begin{lstlisting}[style=Matlab-editor]
n = 5; a = 1; b = 2;
f = @(x) x.^2 - log(x.^2+2);
xvec = zeros(1,n);
for j = 1:n
	xvec(j) = (a+b)/2;
	if f(a)*f(xvec(j)) < 0
		b = xvec(j);
	else
		a = xvec(j);
	end
end
format long
table( (1:n)', xvec', 'VariableNames', {'n', 'x_n'})
format short
	\end{lstlisting}
	\item Study the M-function {\color{blue}naiveFP}, then add comments to the main body of it.
	\item Run the following, then explain why fixed point iteration can be used here.
	\begin{lstlisting}[style=Matlab-editor]
 clear all; close all;
 g = @(x) sqrt(log(x.^2+2));
 xvec = naiveFP(g, 1, 1e-6, 10, 'Numerical solutions to $x^2=\ln(x^2+2)$', 'b--o');
	\end{lstlisting}
	\item The absolute error, or simply error, is defined as
	\[ E_k=|x_k-x^*|, \]
	where \(x^∗\) is the true root.
	The relative error is defined as
	\[ \epsilon_k=\left|\frac{x_k-x^*}{x^*}\right|, \]
	In practice, the true root is not known, we estimate the errors by
	\[ E_k=|x_k-x_{k-1}| \quad \text{and} \quad  \epsilon_k=\left|\frac{x_k-x_{n-1}}{x_{n-1}}\right|\]
	Estimate the errors in part (e).
	Write down your estimates below.
	Based on this empirical data, does the sequence \(\{x_k\}\) converge linearly or quadratically?
	\item Instead of the regular fixed point iteration \(x_{k+1} = g(x_k)\), implement
	\begin{align*}
	\alpha &= g(x_k), \quad \beta = g(\alpha), \\
	x_{k+1}&=x_k-\frac{(\alpha-x_k)^2}{\beta-2\alpha+x_k}\\
		&=\frac{\beta x_k-\alpha^2}{\beta-2\alpha+x_k}, \quad \text{for } k=0,1,\cdots,n
	\end{align*}
	Does this iteration scheme converges fast?
	\item Add a warning message to your code for part (g) for dividing a small number.
\end{enumerate}


\section*{Task 2}
Use what you have learnt in Task 1 to find the root of
\[ f(x)=\tan(x)-(x-0.4)(x-0.7) \]
near \(x = 0\).
Use \(tol= 10^{-10}\) and \(nmax = 50\).


\section*{Task 3}
Consider the function
\[ f(x) = e^x - 4x^2. \]
\begin{enumerate}[(a)]
	\item Plot \(f(x)\) over various intervals to confirm that the function has 3 roots:
	\[ x_1^*\in(-1,0), \quad x_2^*\in(0,1), \quad x_3^*\in(4,4.5). \]
	\item Study and use the M-function \hyperref[bisect]{bisect} to find \(x_1^*\).
	\item Study and use the M-function \hyperref[newton]{newton} to find \(x_1^*\).
	\item Study and use the M-function \hyperref[fixpt]{fixpt} to find \(x_1^*\).
\end{enumerate}


\section*{Task 4}
Newton's method converges slowly when derivative/s are zero at the root.
In assignment 7, there is a question on how to speed up Newton's method when the order \(m\) is known in advance.
However, this information is unlikely to be available in practice.
This question illustrates how to estimate the order \(m\).
Consider
\[ f(x)=(x-2)^3 e^x, \quad x\in[0,3]. \]
\begin{enumerate}[(a)]
	\item Plot \(f(x)\) and its derivatives over the interval \([0, 3]\) to determine the order.
	\item Implement the following iteration scheme for the function
	\[ m_k=\frac{x_{k-1}-x_{k-2}}{2x_{k-1}-x_k-x_{k-2}}, \quad k\geq 2, \]
	where \(\{x_k\}\) are the sequence computed by Newton's method.
	What do you notice?
\end{enumerate}


\section*{Task 5}
In MATLAB\texttrademark\ there exist functions for finding roots of a function: {\color{blue}roots, fzero, fsolve}.
Consider the following function
\[ f(x)=x^2-\sin(x+1). \]
\begin{enumerate}[(a)]
	\item Use the M-function \hyperref[newton]{newton} to find the root/s of \(f\).
	\item Use the M-function \hyperref[secant]{secant} to find the root/s of \(f\).
	\item Study and {\color{blue}fzero} to find the root/s.
\end{enumerate}


\section*{Task 6}
Find \(y\) that satisfies the following equation
\[ 0.7=1-\frac{2}{a}\left[\frac{1}{a}\int_0^a\frac{x}{\exp(x)-1}\ud x+\frac{a}{6}-1 \right]. \]
State and compare your solution with the solution from {\color{blue}fsolve}.
Explain your method.