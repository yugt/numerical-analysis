%!TEX root = ../main.tex
\titleformat{\section}{\normalfont\Large\bfseries}{Task \thesection}{1em}{}
\chapter{Intergration}

\section*{Task 1}
In this question, we will investigate the convergence of some common quadrature rules.
\begin{enumerate}[(a)]
	\item Study and write down the output of the following commands.
	\begin{lstlisting}[style=Matlab-editor]
syms x
fsym = symfun(1/(1+25*x^2), x); %Symbolic f
fsym( -1:1:1 )
fano = @(x) 1./(1+25*x.^2); %Anonymous f
fano( -1:1:1 )
exactf = int(fsym,-1,1);
format longE
eval(exactf)
format short
	\end{lstlisting}
	\item Write an M-function to implement the so-called midpoint quadrature rule
	\[ \int_{a}^{b}f(x)\ud x\approx \sum_{k=1}^n(x_k-x_{k-1})f\left(\frac{x_{k-1}+x_k}{2}\right),\]
	where \(a=x_0\) and \(b=x_n\).
	The code for you m-file might look like the following:
	\begin{lstlisting}[style=Matlab-editor]
function quad = midpointquad( f, a, b, n)
%---------------------------------------------
%MIDPOINTQUAD Midpoint Quadrature Rule
% Input
% f integrand
% a, b upper and lower limits of integration
% n the number of equally spaced subintervals
% Output
% quad quadrature value
%
% If f is defined as an M-function use the @ notation to call
% quad=midpointquad(@f,a,b,n)
% If f is defined as an anonymous/symbolic function then simply call
% quad=midpointquad(f,a,b,n)
%---------------------------------------------
xpts = linspace( ??? ) ;
h = ??? ;
xmidpts = zeros(1, ??? );
for i = 1: ???
	xmidpts(i) = ??? ;
end
quad = h * sum ( ??? );
	\end{lstlisting}
	\item Run the following commands and study the output of it.
	\begin{lstlisting}[style=Matlab-editor]
N = 5;
nvec = (2 * 10 .^ (1:1:N));
result = zeros(1, N);
error = zeros(1,N);
for i = 1:N
	result(i) = midpointquad(fano,-1,1,nvec(i));
	error(i) = eval(result(i) - exactf);
end
format longE
T = table( categorical(nvec'), categorical(2./nvec'), result', error', 'VariableNames', {'n' 'h' 'Midpoint' 'Error'})
format short
	\end{lstlisting}
	Use the output to estimate the convergence rate of the midpoint rule in terms of \(h\).
	\item Write an M-function to implement the so-called trapezoidal quadrature rule
	\[ \int_{a}^{b}f(x)\ud x\approx \sum_{k=1}^n(x_k-x_{k-1})\frac{f(x_{k-1})+f(x_k)}{2}, \]
	using equally spaced subintervals, where \(a=x_0\) and \(b=x_n\).
	Name this M-function \texttt{trapezoidalquad}, then estimate the convergence rate of the trapezoidal rule in terms of h using Runge's function.
	\item Write an M-function to implement the so-called Simpson's quadrature rule
	\[ \int_{a}^{b}f(x)\ud x\approx \sum_{k=1}^n\frac{x_k-x_{k-1}}{6}\left[ f(x_{k-1})+4\left(\frac{x_{k-1}+x_k}{2}\right)+f(x_k)\right], \]
	using equally spaced subintervals.
	Name this M-function \texttt{simpsonquad}, then estimate the convergence rate of the Simpson's rule in terms of \(h\) using Runge's function.
	\item Consider applying the midpoint quadrature rule to the following integral
	\[ \int_0^1\ln x\ud x=-1. \]
	Estimate the rate of convergence of the midpoint quadrature rule using this integral.
\end{enumerate}


\section*{Task 2}
In this question, we will test Newton-Cotes, Clenshaw-Curtis, Gaussian quadrature rules.
\begin{enumerate}[(a)]
	\item Use the given M-function \hyperref[newpoly]{newpoly} to construct interpolating polynomials in Newton's form of degree \(n=1,2,3,4,5,10,15,20\) on equally spaced nodes for Gaussian function
	\[ f(x)=e^{x^2} \]
	then integrate the polynomials symbolically over \([-1, 1]\) to approximate
	\[ \int_{-1}^{1}e^{x^2}\ud x. \]
	Note this approach does NOT break \([-1, 1]\) into subintervals.
	\item Repeat part (a), but this time, do it for Runge's function
	\[ \int_{-1}^{1}\frac{1}{1+25x^2}\ud x. \]
	\item Repeat part (b), but this time, do it using Chebyshev nodes instead.
	Study and compare the accuracy of it with the approximation in part (b).
	\item Repeat part (b), but this time, do it using the given M-function \hyperref[fclencurt]{fclencurt} 	and \hyperref[lgwt]{lgwt}, which output the Clenshaw-Curtis and Gauss-Legendre nodes and weights, respectively.
	Compare the accuracy of those two quadrature rules with the approximations in part (b) and part (c).
\end{enumerate}



\section*{Task 3}
This question is to investigate the so-called {\color{blue}degree of exactness} of a quadrature rule.
\begin{enumerate}[(a)]
	\item Run the following commands and study the output of it
	\begin{lstlisting}[style=Matlab-editor]
clear all
syms x
N = 5;
miderror = zeros(1,N);
traperror = zeros(1,N);
simerror = zeros(1,N);
for i = 1:N
	fsym = symfun(i*x^(i-1), x);
	miderror(i) = midpointquad(fsym,0,1,1) - 1;
	traperror(i) = trapezoidalquad(fsym,0,1,1) - 1;
	simerror(i) = simpsonquad(fsym,0,1,1) - 1 ;
end
format long
monomial = {'$1$';'$2x$';'$3x^2$';'$4x^3$';'$5x^4$'};
T = table(monomial , miderror', traperror', simerror', 'VariableNames', {'func' 'MidpointError' 'TrapezoidalError' 'SimpsonError' })
format short
	\end{lstlisting}
	What can you conclude from the output?
	\item Do a similar investigation for Clenshaw-Curtis and Gauss-Legendre.
\end{enumerate}


\section*{Task 4}
This question is to combine and extend what we have learnt about quadrature rules so far.
\begin{enumerate}
	\item Based on Task 1-3, propose an algorithm by writing an M-function that
performs a numerical integration, and test it using the following integral
	\[ \int_{-1}^{1}\frac{1}{1+25x^2}\ud x. \]
	\item  Suggest two additional techniques that can be used to numerically solve the
following integrals, or integrals that have similar features.
	\[ \int_{0}^{\infty}\frac{1}{1+25x^2}\ud x \quad \text{and} \quad \int_{-1}^{1}\sqrt{\left|x-\frac{1}{2}\right|}\ud x . \]

\end{enumerate}