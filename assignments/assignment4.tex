%!TEX root = ../main.tex
\chapter{Orthogonal Polynomials and Interpolation}

\section{Legendre polynomials}
Let \((Q_n)_{n\in\N}\) be the sequence of polynomials defined by \(Q_n(x) = \frac{1}{2^n n!} \left((x^2-1)^n\right)^{(n)} \).
This sequence defines the Legendre polynomials.
\begin{enumerate}
	\item Using the constant weight function \(w(x) = 1\) over \((-1, 1)\), show that \((Q_n)_{n\in\N}\) defines a sequence of orthogonal polynomials.
	\begin{proof}
	First, we want to prove the inner product of any Legendre polynomials with themselves are non-zero.
	Define
	\[ I_n=\int_{-1}^{1}Q_n^2(x)\ud x \]
	We can easily get the first Legendre polynomial
	\[ Q_0(x)=1, \quad I_0=\int_{-1}^{1}\ud x=2 \]
	So \(I_0\neq0\) trivially holds for \(n=0\).
	By integration by parts,
	\begin{align*}
	I_n&=\frac{1}{2^{2n}(n!)^2} \int_{-1}^{1} \D{n}{x}\left[\left(x^2-1\right)^n \right] \cdot \D{n}{x}\left[\left(x^2-1\right)^n \right] \ud x\\
	&=\frac{1}{2^{2n}(n!)^2} \int_{-1}^{1} \D{n}{x}\left[\left(x^2-1\right)^n \right] \cdot \ud \left\{\D{n-1}{x}\left[\left(x^2-1\right)^n \right]\right\}\\
	&=\frac{1}{2^{2n}(n!)^2} \left\{ \left.\D{n}{x}\left[\left(x^2-1\right)^n \right]\D{n-1}{x}\left[\left(x^2-1\right)^n \right]\right|_{-1}^{1}\right. \\
	& \left. - \int_{-1}^{1} \D{n-1}{x}\left[\left(x^2-1\right)^n \right] \cdot \D{n+1}{x}\left[\left(x^2-1\right)^n \right] \ud x \right\}
	\end{align*}
	{\color{red}Calculation Mistakes}
	Since the \(\D{k}{x}\left[\left(x^2-1\right)^n \right]\) always has a factor \(\left(x^2-1\right)\) for \(0\leq k\leq n\), the first term is always zero, which means
	\[ I_n=\frac{-1}{2^{2n}(n!)^2} \int_{-1}^{1} \D{n-1}{x}\left[\left(x^2-1\right)^n \right] \cdot \D{n+1}{x}\left[\left(x^2-1\right)^n \right] \ud x \]
	By repeating $n$ times of integration by part, we get
	\[ I_n=\frac{(-1)^n}{2^{2n}(n!)^2} \int_{-1}^{1} \left[\left(x^2-1\right)^n \right] \cdot \D{2n}{x}\left[\left(x^2-1\right)^n \right] \ud x \]
	Expand \(\left(x^2-1\right)^n\) by binomial theorem, and only \(x^{2n}\) has a nonzero \(2n^\textrm{th}\) derivative.
	\[ \D{2n}{x}\left[\left(x^2-1\right)^n \right]=(2n)! \]
	Plug in this result and
	\[ I_n=\frac{(2n)!}{2^{2n}(n!)^2} \int_{-1}^{1} \left(1-x^2\right)^n \ud x \]
	Define the integral part as
	\[ K_n=\int_{-1}^{1} \left(1-x^2\right)^n \ud x \]
	Integration by parts (taking a factor 1 as the second part) gives
	\[ K_n=\int_{-1}^{1} 2nx^2\left(1-x^2\right)^{n-1}\ud x \]
	Writing \(2nx^2\) as \(2n-2n\left(1-x^2\right)\) we obtain
	\begin{align*}
	K_n&=2n\int_{-1}^{1} \left(1-x^2\right)^{n-1} \ud x -2n\int_{-1}^{1} \left(1-x^2\right)^n \ud x\\
	&=2n K_{n-1} -2n K_{n}
	\end{align*}
	So we get the recursive relation
	\[ (2n+1)K_n = 2n K_{n-1} \]
	Expend this relation down to $K_0$, we have
	\[ K_n=\frac{2n}{2n+1}\cdot\frac{2n-2}{2n-1}\cdots\frac{2}{3}\cdot K_0 \]
	It is easy to get \(K_0=2\), and reorganizing the terms,
	\[ K_n=\frac{2^{2n+1}(n!)^2}{(2n+1)!} \]
	By substitution and some simple calculation,
	\[ I_n=K_n\cdot\frac{(2n)!}{2^{2n}(n!)^2}=\frac{2}{2n+1} \]

	Then, we need to prove the Legendre polynomials are mutually orthogonal.
	We introduce the following property of Legendre polynomials
	\begin{theorem}[Legendre differential equation]
	Legendre polynomials are solutions to the Legendre differential equation
	\[ \left[\left(1-x^2\right)Q_n'(x)\right]'+n(n+1)Q_n(x)=0 \]
	\end{theorem}
	\begin{proof}
	To prove that Legendre polynomials solve the Legendre differential equation, we need the differential rule for a product of two factors.
	\begin{theorem}[General Leibniz rule]
	$f$ and $g$ are $n$-times differentiable functions, then the product $fg$ is also $n$-times differentiable and its \(n^\textrm{th}\) derivative is given by
	\[ \D{n}{x}[f(x)g(x)]=\sum_{i=0}^{n}\binom{n}{i}\D{n-i}{x}f(x)\cdot \D{i}{x}g(x) \]
	where the zeroth derivative is defined to be the function itself.
	\end{theorem}
	\begin{proof}
	We prove by induction.
	At \(n=1\), we get
	\[ (fg)'(x)=f'(x)g(x)+f(x)g'(x) \]
	This is the basic Leibniz rule, so the equality holds for \(n=1\).
	We assume that the equality
	\[(fg)^{(m)}(x)=\sum_{k=0}^m{\binom{m}{k}}f^{(m-k)}(x)g^{(k)}(x)\]
	Therefore, at $m+1$ we get:
	\begin{align*}
		(fg)^{(m+1)}(x)&=\frac{\text{d}}{\text{d}x}\sum_{k=0}^{m}{\binom{m}{k}}f^{(m-k)}(x)g^{(k)}(x)\\
		&=\sum_{k=0}^{m}{\binom{m}{k}}f^{(m-k)}(x)g^{(k+1)}(x)+\sum_{k=0}^{m}{\binom{m}{k}}f^{(m+1-k)}(x)g^{(k)}(x)\\
		&=\sum_{k=1}^{m+1}{\binom{m}{k-1}}f^{(m+1-k)}(x)g^{(k)}(x)+\sum_{k=0}^{m}{\binom{m}{k}}f^{(m+1-k)}(x)g^{(k)}(x)\\
		&={\binom{m}{m}}f(x)g^{(m+1)}(x)+\sum_{k=1}^{m}{\binom{m}{k-1}}f^{(m+1-k)}(x)g^{(k)}(x)\\
		&+\sum_{k=1}^{m}{\binom{m}{k}}f^{(m+1-k)}(x)g^{(k)}(x)+{\binom{m}{0}}f^{(m+1)}(x)g(x)\\
		&={\binom{m}{m}}f(x)g^{(m+1)}(x)+\sum_{k=1}^{m}\left[{\binom{m}{k-1}}+{\binom{m}{k}}\right]f^{(m+1-k)}(x)g^{(k)}(x)\\
		&+{\binom{m}{0}}f^{(m+1)}(x)g(x)\\
		&={\binom{m+1}{m+1}}f(x)g^{(m+1)}(x)+\sum_{k=1}^m{\binom{m+1}{k}}f^{(m+1-k)}(x)g^{(k)}(x)\\
		&+{\binom{m+1}{0}}f^{(m+1)}(x)g(x)\\
		&=\sum_{k=0}^{m+1}{\binom{m+1}{k}}f^{(m+1-k)}(x)g^{(k)}(x).
	\end{align*}
	By principle of mathematical induction, this holds for all \(n\in\N^*\).
	\end{proof}
	To prove that Legendre polynomials are solution to Legendre differential equation, let \(y=\left(x^2-1\right)^n\), so that \sloppy \(y'=2nx\left(x^2-1\right)^{n-1}\), and 
	\[ \left(x^2-1\right)y'-2nxy=0 \]
	Differentiate this expression \(n+1\) times using the general Leibniz rule, we obtain
	\begin{small}
		\[ \left(x^2-1\right)\DD{n+2}{x}{y}+2x(n+1)\DD{n+1}{x}{y}+n(n+1)\DD{n}{x}{y}-2n\left[x\DD{n+1}{x}{y}+(x+1)\DD{n}{x}{y}\right]=0 \]
	\end{small}
	which reduces to 
	\[ \left(x^2-1\right)\D{n+2}{x}y+2x\D{n+1}{x}y-n(n+1)\D{n}{x}y=0 \]
	Dividing each side by \(-2^n(n!)\), we get
	\[ \left(1-x^2\right)\D{n+2}{x}\frac{y}{2^n(n!)}+2x\D{n+1}{x}\frac{y}{2^n(n!)}-n(n+1)\D{n}{x}\frac{y}{2^n(n!)}=0 \]
	Notice that
	\[ Q_n(x)=\D{n}{x}\frac{y}{2^n(n!)} \]
	Apply the substitution, we obtain
	\[ \left(1-x^2\right)Q_n''(x)+2xQ_n'(x)-n(n+1)Q_n(x)=0 \]
	which is exactly
	\[ \left[\left(1-x^2\right)Q_n'(x)\right]'+n(n+1)Q_n(x)=0 \]
	\end{proof}
	Now we can prove that two different Legendre polynomials are mutually orthogonal.
	With the Legendre differential equation, we multiply both sides by \(Q_k(x)\), where \(n\neq k\), and integrate it in \([-1,1]\).
	\begin{equation}
	\label{eqn4.1}
	\int_{-1}^{1} Q_k(x)\left[\left(1-x^2\right)Q_n'(x)\right]'\ud x + \int_{-1}^{1}Q_k(x)n(n+1)Q_n(x)\ud x=0
	\end{equation}
	We also have the Legendre differential equation for \(Q_k(x)\):
	\[ \left[\left(1-x^2\right)Q_k'(x)\right]'+k(k+1)Q_k(x)=0 \]
	Multiply both sides by \(Q_n(x)\), where \(n\neq k\), and integrate it in \([-1,1]\), we get a dual form
	\begin{equation}
	\label{eqn4.2}
	\int_{-1}^{1} Q_n(x)\left[\left(1-x^2\right)Q_k'(x)\right]'\ud x + \int_{-1}^{1}Q_n(x)k(k+1)Q_k(x)\ud x=0.
	\end{equation}
	Integrate by part on first term of both (\ref{eqn4.1}) and (\ref{eqn4.2}), we get
	\[\begin{cases}
	-\int_{-1}^{1}Q_k'(x)\left[\left(1-x^2\right)Q_n'(x)\right]\ud x+\int_{-1}^{1}Q_k(x)n(n+1)Q_n(x)\ud x=0\\
	-\int_{-1}^{1}Q_n'(x)\left[\left(1-x^2\right)Q_k'(x)\right]\ud x+\int_{-1}^{1}Q_n(x)k(k+1)Q_k(x)\ud x=0
	\end{cases}\] 
	Notice that the first terms of the two equations are equal, we have the second terms equal, which means
	\[ n(n+1)\int_{-1}^{1}Q_k(x)Q_n(x)\ud x=k(k+1)\int_{-1}^{1}Q_n(x)Q_k(x)\ud x \]
	Since we are considering different Legendre polynomials, we have \(n\neq k\), so \(n(n+1)\neq k(k+1)\).
	Then we conclude that
	\[ \int_{-1}^{1}Q_n(x)Q_k(x)\ud x=0 \]
	So the mutual orthogonality is also proved.
	\end{proof}
	\item Show that \(Q_n(-x)=(-1)^n Q_n(x)\).
	\begin{proof}
	We expand \(\left(x^2-1\right)^n\) by binomial theorem.
	\[ \left(x^2-1\right)^n=\sum_{i=0}^{n}\binom{n}{i}x^{2n}(-1)^{n-i} \]
	Split the cases where $n$ is even and $n$ is odd.
	\begin{align*}
	\D{2m+1}{x}\left(x^2-1\right)^{2m+1}&=\sum_{i=m+1}^{2m+1}\binom{2m+1}{i}\frac{(2i)!}{(2i-2m-2)!}x^{2i-2m-1}(-1)^{2m+1-i}\\
	\D{2m}{x}\left(x^2-1\right)^{2m}&=\sum_{i=m}^{2m}\binom{2m}{i}\frac{(2i)!}{(2i-2m)!}x^{2i-2m}(-1)^{2m-i}
	\end{align*}
	Combine these two results,
	\[ Q_n(x)=\begin{cases} \frac{1}{2^{2m+1}(2m+1)!} \sum_{i=m+1}^{2m+1}\binom{2m+1}{i}\frac{(2i)!}{(2i-2m-2)!}x^{2(i-m)-1}(-1)^{2m+1-i} & n=2m+1 \\
	\frac{1}{2^{2m}(2m)!}\sum_{i=m}^{2m}\binom{2m}{i}\frac{(2i)!}{(2i-2m)!}x^{2(i-m)}(-1)^{2m-i} & n=2m \end{cases} \]
	We can see that when \(n=2m\), \(Q_n(x)\) is a sum of \(x^{2k}\) with some coefficients, which means it is an even function.
	When \(n=2m+1\), \(Q_n(x)\) is a sum of \(x^{2k+1}\) with some coefficients, which means it is an odd function.
	Since \(Q_n(x)\) is even when $n$ is even, and is odd when $n$ is odd, we have
	\[ Q_n(-x)=(-1)^n Q_n(x) \]
	\end{proof}
	\item Show that for any \(x \in [-1, 1]\), the Legendre polynomials obey the recurrence relation
	\[ (n + 1)Q_{n+1}(x) = (2n + 1)x Q_n(x) - n Q_{n-1}(x) \]
	\begin{proof}
	This can be proved with pure calculation, with the application of general Leibniz rule.
	\begin{align*}
		(n+1)Q_{n+1}(x)&=\frac{n+1}{2^{n+1}(n+1)!}\D{n+1}{x}\left[\left(x^2-1\right)^{n+1}\right]\\
		&=\frac{1}{2^{n+1}n!}\D{n}{x}\left[2x(n+1)\left(x^2-1\right)^{n}\right]\\
		&=\frac{n+1}{2^{n}n!}\D{n}{x}\left[x\left(x^2-1\right)^{n}\right]
	\end{align*}
	Apply the general Leibniz rule on the differential,
	\begin{equation}
	\label{eqn4.3}
	\D{n}{x}\left[x\left(x^2-1\right)^{n}\right]=x\D{n}{x}\left[\left(x^2-1\right)^{n}\right]+n\D{n-1}{x}\left[\left(x^2-1\right)^{n}\right]
	\end{equation}
	Apply the substitution to get
	\begin{align*}
	(n+1)Q_{n+1}(x)&=\frac{n+1}{2^{n}n!}x\D{n}{x}\left[\left(x^2-1\right)^{n}\right]+\frac{n+1}{2^{n}n!}n\D{n-1}{x}\left[\left(x^2-1\right)^{n}\right]\\
	&=x(n+1)Q_n(x)+\frac{n(n+1)}{2^{n}n!}\D{n-1}{x}\left[\left(x^2-1\right)^{n}\right]\\
	&=x(2n+1)Q_n(x)-nxQ_n(x)-\frac{n}{2^n n!}x\D{n}{x}\left[\left(x^2-1\right)^{n}\right]
	\end{align*}
	The last term without its coefficient is already in (\ref{eqn4.3}), so
	\[ x\D{n}{x}\left[\left(x^2-1\right)^{n}\right]=\D{n}{x}\left[x\left(x^2-1\right)^{n}\right]-n\D{n-1}{x}\left[\left(x^2-1\right)^{n}\right] \]
	Apply substitution,
	\begin{align*}
	(n+1)Q_{n+1}(x)&=x(2n+1)Q_n(x)+\frac{n+1}{2^n (n-1)!}\D{n-1}{x}\left[\left(x^2-1\right)^{n}\right]\\
	&-\frac{n}{2^n n!}\D{n}{x}\left[x\left(x^2-1\right)^{n}\right]+\frac{n}{2^n n!}\D{n-1}{x}\left[n\left(x^2-1\right)^{n}\right]\\
	&=x(2n+1)Q_n(x)+\frac{1}{2^n(n-1)!}\\
	&\cdot\D{n-1}{x}\left\{(n-1)\left(x^2-1\right)^{n}-\frac{\ud}{\ud x}\left[x\left(x^2-1\right)^{n}\right]+n\left(x^2-1\right)^{n} \right\}\\
	&=x(2n+1)Q_n(x)\\
	&+\frac{1}{2^n(n-1)!}\D{n-1}{x}\left[2n\left(x^2-1\right)\left(x^2-1\right)^{n-1}-2x^2n\left(x^2-1\right)^{n-1} \right]\\
	&=x(2n+1)Q_n(x)-\frac{2n}{2^n(n-1)!}\D{n-1}{x}\left[\left(x^2-1\right)^{n}\right]\\
	&=x(2n+1)Q_n(x)-nQ_{n-1}(x)
	\end{align*}
	This proves the recurrence relation of Legendre polynomials.
	\end{proof}
	\item Prove that
	\[ Q_n(x)=\sum_{i=0}^{n}(-1)^i \binom{n}{i}^2 \left(\frac{1+x}{2}\right)^{n-i} \left(\frac{1-x}{2}\right)^{i} \]
	\begin{proof}
	This is a direct result from the general Leibniz rule.
	\[ Q_n(x)=\frac{1}{2^nn!}\D{n}{x}\left[\left(x-1\right)^n\left(x+1\right)^n\right] \]
	Apply the general Leibniz rule, where
	\[ f(x)=(x-1)^n \quad g(x)=(x+1)^n \]
	And their higher order derivatives are
	\begin{align*}
	\D{n-i}{x}f(x)&=n(n-1)\cdots(i+1)(x-1)^i=\frac{n!}{i!}(x-1)^i\\
	\D{i}{x}g(x)&=n(n-1)\cdots(n-i+1)(x-1)^i=\frac{n!}{(n-i)!}(x+1)^i
	\end{align*}
	Plug these two results in the binomial expansion of Legendre polynomials,
	\begin{align*}
	Q_n(x)&=\frac{1}{2^nn!}\sum_{i=0}^{n}\binom{n}{i}\frac{n!}{i!}(x-1)^i\frac{n!}{(n-i)!}(x+1)^{n-i}\\
	&=2^{-n}\sum_{i=0}^{n}\binom{n}{i}^2(x-1)^i(x+1)^{n-i}\\
	&=\sum_{i=0}^{n}(-1)^i \binom{n}{i}^2 \left(\frac{1+x}{2}\right)^{n-i} \left(\frac{1-x}{2}\right)^{i}
	\end{align*}
	This proves the result.
	\end{proof}
\end{enumerate}


\section{Interpolation}
Let $f$ be a continuous function for which we know the eight points defined in the following table.
\begin{center}
	\begin{tabular}{ccccccccc}
	\hline
	$x$ & $-5$ & $-1$ & 0 & 1 & 3 & 5 & 10 & 10 \\ \hline
	$f(x)$ & 781 & 5 & 1 & 1 & 61 & 521 & 9091 & 19141 \\ \hline
	\end{tabular}
\end{center}
Determine $f(2)$.
\begin{proof}[Answer]
We apply polynomial interpolation.
Since there are 8 distinct points to pass through, we need a polynomial with degree at least 7.
Assume the polynomial has form
\[ p(x)=a_0 + a_1 x + a_2 x^2 + a_3 x^3 + a_4 x^4 + a_5 x^5 + a_6 x^6 + a_7 x^7 \]
Then with the given points we have the following system of linear equations
\[\begin{bmatrix}
x_1^0 & x_1^1 & x_1^2 & x_1^3 & x_1^4 & x_1^5 & x_1^6 & x_1^7 \\
x_2^0 & x_2^1 & x_2^2 & x_2^3 & x_2^4 & x_2^5 & x_2^6 & x_2^7 \\
x_3^0 & x_3^1 & x_3^2 & x_3^3 & x_3^4 & x_3^5 & x_3^6 & x_3^7 \\
x_4^0 & x_4^1 & x_4^2 & x_4^3 & x_4^4 & x_4^5 & x_4^6 & x_4^7 \\
x_5^0 & x_5^1 & x_5^2 & x_5^3 & x_5^4 & x_5^5 & x_5^6 & x_5^7 \\
x_6^0 & x_6^1 & x_6^2 & x_6^3 & x_6^4 & x_6^5 & x_6^6 & x_6^7 \\
x_7^0 & x_7^1 & x_7^2 & x_7^3 & x_7^4 & x_7^5 & x_7^6 & x_7^7 \\
x_8^0 & x_8^1 & x_8^2 & x_8^3 & x_8^4 & x_8^5 & x_8^6 & x_8^7
\end{bmatrix}
\begin{bmatrix}
a_0 \\ a_1 \\ a_2 \\ a_3 \\ a_4 \\ a_5 \\ a_6 \\ a_7
\end{bmatrix}=
\begin{bmatrix}
f(x_0) \\ f(x_1) \\ f(x_2) \\ f(x_3) \\ f(x_4) \\ f(x_5) \\ f(x_6) \\ f(x_7)
\end{bmatrix}\]
With numerical value,
\[\begin{bmatrix}
1 &	-5	&	25	&	-125	&	625		&	-3125	&	15625	&	-78125	\\
1 &	-1	&	1	&	-1		&	1		&	-1		&	1		&	-1		\\
1 &	0	&	0	&	0		&	0		&	0		&	0		&	0		\\
1 &	1	&	1	&	1		&	1		&	1		&	1		&	1		\\
1 &	3	&	9	&	27		&	81		&	243		&	729		&	2187	\\
1 &	5	&	25	&	125		&	625		&	3125	&	15625	&	78125	\\
1 &	10	&	100	&	1000	&	10000	&	100000	&	1000000	&	10000000\\
1 &	12	&	144	&	1728	&	20736	&	248832	&	2985984	&	35831808
\end{bmatrix}
\begin{bmatrix}
a_0 \\ a_1 \\ a_2 \\ a_3 \\ a_4 \\ a_5 \\ a_6 \\ a_7
\end{bmatrix}=
\begin{bmatrix}
781 \\ 5 \\ 1 \\ 1 \\ 61 \\ 521 \\ 9091 \\ 19141
\end{bmatrix}\]
Write it in the matrix form,
\[ Xa=f \]
Then, we can solve for $a$ as
\[ a=X^{-1}f, \quad \textrm{or in MATLAB\texttrademark\ syntax, } a=X\backslash f \]
By MATLAB\texttrademark, we get
\[ a=(1,-1,1,-1,1,0,0,0)^T \]
So the interpolating polynomial is
\[ p(x)=1-x+x^2-x^3+x^4 \]
Plug in $x=2$, we get the estimate
\[ f(2) \approx p(2) = 11 \]
The MATLAB\texttrademark\ code is attached here.
\begin{lstlisting}[style=Matlab-editor]
x = [-5 -1 0 1 3 5 10 12];
X = [x.^0;x.^1;x.^2;x.^3;x.^4;x.^5;x.^6;x.^7]';
y = [781 5 1 1 61 521 9091 19141]';
X\y
\end{lstlisting}
\end{proof}


\section{Newton's form of the interpolation polynomial}
Let \(f\) be a continuous function, \(P^n\) be its interpolation polynomial in the points  \(x_0\), \(\cdots\), \(x_n\).
\begin{enumerate}
	\item Let \(P^0(x)=f(x_0)\) be the interpolation polynomial in a single point $x_0$.
	\begin{enumerate}[(a)]
		\item Show that for two points $x_0$ and $x_1$, \(P^1(x)=P^0(x)+\frac{f(x_1)-f(x_0)}{x_1-x_0}(x-x_0)\).
		\begin{proof}
		We evaluate \(P^1(x)\) at $x_0$ and $x_1$.
		\begin{align*}
		P^1(x_0)&=f(x_0)+\frac{f(x_1)-f(x_0)}{x_1-x_0}(x_0-x_0)=f(x_0)\\
		P^1(x_1)&=f(x_0)+\frac{f(x_1)-f(x_0)}{x_1-x_0}(x_1-x_0)=f(x_0)+f(x_1)-f(x_0)=f(x_1)
		\end{align*}
		Obviously, \(P^1(x)\) is a polynomial with degree of at most 1, so it represents a straight line, and two distinct points are enough to determine a unique \(P^1(x)\).
		\end{proof}
		\item Determine a polynomial $R$ of degree at most two, such that \( P^2(x)=P^1(x)+R(x) \), for three nodes $x_0$, $x_1$ and $x_2$.
		\begin{proof}
		Since \(P^2(x)\) is an interpolation polynomial in the points \(x_0,x_1,x_2\), we have \(P^2(x_2)=f(x_2)\).
		Since \(P^1(x)\) has already satisfied \(P^1(x_0)=f(x_0)\) and \(P^1(x_1)=f(x_1)\), we need \(R(x_0)=R(x_1)=0\).
		Thus, we construct \(R(x)\) to have zeros at $x_0$ and $x_1$, with an undetermined coefficient $a_2$ to pass through \((x_2,f(x_2))\).
		\[ P^2(x)=P^1(x)+a_2(x-x_0)(x-x_1) \]
		With boundary condition, we have
		\[ f(x_2)=P^1(x_2)+a_2(x_2-x_0)(x_2-x_1) \]
		So we can solve for $a_2$
		\[ a_2=\frac{f(x_2)-P^1(x_2)}{(x_2-x_0)(x_2-x_1)} \]
		Hence, we obtain
		\[ R(x)=\frac{f(x_2)-P^1(x_2)}{(x_2-x_0)(x_2-x_1)}(x_2-x_0)(x_2-x_1) \]
		\end{proof}
		\item Prove by induction that
		\[ P^j(x)=P^{j-1}(x)+a_j\prod_{k=0}^{j-1}(x-x_k), \quad \textrm{ where } a_j \textrm{ only depends on } x_0,\cdots x_j \]
		\begin{proof}
		For \(j=1\) and \(j=2\), we have already proved in the previous steps.
		Assume this holds for \(j=n\), which means for all $m$ such that \(0\leq m\leq n, P^n(x_m)=f(x_m)\).
		We need to prove
		\[ \forall m \textrm{ such that } 0\leq m\leq n+1, P^{n}(x_m)+a_{n+1}\prod_{k=0}^{n}(x_m-x_k) = f(x_m)  \]
		This is obvious for \(0\leq m\leq n\), since the product always contains a zero factor and is zero.
		So we just need the condition for \(m=n+1\).
		\[ P^{n}(x_{n+1})+a_{n+1}\prod_{k=0}^{n}(x_{n+1}-x_k) = f(x_{n+1}) \]
		Then we obtain the coefficient
		\[ a_{n+1}=\frac{f(x_{n+1})-P^{n}(x_{n+1})}{\prod_{k=0}^{n}(x_{n+1}-x_k)} \]
		This only depends on \(x_0,\cdots,x_{n+1}\).
		So we have
		\[ P^{n+1}(x):=P^n(x)+\frac{f(x_{n+1})-P^{n}(x_{n+1})}{\prod_{k=0}^{n}(x_{n+1}-x_k)}\prod_{k=0}^{n}(x-x_k) \]
		which interpolates \(f(x)\) in \(x_0,\cdots,x_{n+1}\).
		This is a polynomial with degree at most $n+1$.
		By invertibility of Vandermonde matrix, we know that $n+1$ distinct points can uniquely determine a polynomial with degree at most $n+1$.
		So this definition exactly define the interpolation polynomial of \(f(x)\).
		By the principle of mathematical induction, we have it true for all \(j\in\N^*\).
		\end{proof}
	\end{enumerate}
	\item Denoting $a_j$ by \(f[x_0,\cdots, x_j]\), show that
	\[ P^n(x)=f(x_0)+\sum_{j=1}^{n} f[x_0,\cdots, x_j] \prod_{k=0}^{j-1}(x-x_k) \]
	\begin{proof}
	From the previous proof, we have already proved that
	\[ P^n(x)=P^{n-1}(x)+a_n\prod_{k=0}^{n-1}(x-x_k) \]
	With this recursive relation, we can prove the proposition by induction.
	For $j=0$, it has been proved that
	\[ P^0(x)=f(x_0) \]
	Assume it is true for \(j=n\), then for $n+1$,
	\[ P^{n+1}(x)=P^{n}(x)+a_{n+1}\prod_{k=0}^{n}(x-x_k) \]
	By induction assumption,
	\[ P^n(x)=f(x_0)+\sum_{j=1}^{n} f[x_0,\cdots, x_j] \prod_{k=0}^{j-1}(x-x_k) \]
	Apply the substitution,
	\begin{align*}
	P^{n+1}(x)&=f(x_0)+\sum_{j=1}^{n} f[x_0,\cdots, x_j]\prod_{k=0}^{j-1}(x-x_k)+a_{n+1}\prod_{k=0}^{n}(x-x_k)\\
	&=f(x_0)+\sum_{j=1}^{n+1} f[x_0,\cdots, x_j]\prod_{k=0}^{j-1}(x-x_k)
	\end{align*}
	By the principle of mathematical induction, we have it true for all \(j\in\N^*\).
	\end{proof}
	\item Prove that for any \(k\in\N^*\),
	\[ \begin{cases} f[x_k] =f(x_k) \\ f[x_0,\cdots, x_k]=\frac{f[x_1,\cdots, x_k]-f[x_0,\cdots, x_{k-1}]}{x_k-x_0} \end{cases} \]
	\begin{proof}
	We prove by induction.
	For \(n=1\), we have proved in the previous step that 
	\[ f[x_0,x_1]=\frac{f(x_1)-f(x_0)}{x_1-x_0} \]
	Suppose this is true for \(k=n\), then according to the previous result, we have the interpolating polynomial with degree at most $n$ passing \((x_0,f(x_0))\), \((x_1,f(x_1))\), \(\cdots\), \((x_{n},f(x_{n}))\)
	\[ P_1^n(x)=f(x_0)+\sum_{j=1}^{n} f[x_0,\cdots, x_j] \prod_{k=0}^{j-1}(x-x_k) \]
	When \(k=n+1\), we also have another polynomial of degree at most $n$ interpolating \sloppy \((x_1,f(x_1)),(x_2,f(x_2)), \cdots, (x_{n+1},f(x_{n+1}))\)
	\[ P_2^n(x)=f(x_1)+\sum_{j=2}^{n+1} f[x_1,\cdots, x_j] \prod_{k=1}^{j}(x-x_k) \]
	We can express \(P^{n+1}(x)\) by these two interpolating polynomials
	\[ P^{n+1}(x)=\frac{x-x_0}{x_{n+1}-x_0}P_2^n(x)+\frac{x_{n+1}-x}{x_{n+1}-x_0}P_1^n(x) \]
	This is constructed to satisfy
	\[ P^{n+1}(x_i)=\begin{cases}P_1^n(x_0)=f(x_0) & i=0 \\ f(x_i) & i=1,\cdots,n \\ P_2^n(x)=f(x_{n+1}) & i=n+1 \end{cases} \]
	So this is indeed the interpolating polynomial of these $n+2$ points.
	And the coefficient of \(x^{n+1}\) is
	\[ \frac{f[x_1,\cdots, x_{n+1}]-f[x_0,\cdots, x_n]}{x_{n+1}-x_0} \]
	By the principle of mathematical induction, we have it true for all \(k\in\N^*\).
	\end{proof}
	\item Write the pseudocode of a clear algorithm to compute \(P^n(x)\) when given $n+1$ nodes \(x_0,\cdots,x_n\) and the value of $f$ at those nodes.
	\ifnum\webview=1
		\begin{algorithm2e}[H]
	\else
		\begin{algorithm2e}[htbp]
	\fi
	\KwIn{$n+1$ nodes: \((x_0,f(x_0)),\cdots,(x_n,f(x_n))\), \(x_0,\cdots,x_n\) stored in vector $x$ and  \(f(x_0),\cdots,f(x_n)\) stored in vector $y$}
	\KwOut{Coefficients of \(P^n(x)\): \(a_0,\cdots,a_n\), stored in vector $a$}
	A=zeros(n+1,n+1)\;
	\For{i=1:n+1}
	{
		\(A(i,1)=y(i)\)\;
	}
	\(a(1)=A(1,1)\)\;
	\For{j=2:n+1}
	{
		\For{i=j:n+1}
		{
			\(A(i,j)=\frac{A(i-1,j-1)-A(i,j-1)}{x(i-j+1)-x(i)}\)\;
		}
		\(a(j)=A(j,j)\)\;
	}
	\caption{Newton's interpolation}
	\end{algorithm2e}
\end{enumerate}

We now consider the case of equidistant nodes, \textit{i.e.}, \(x_i = x_0 + ih\), for any \(0\leq i\leq n\), and some \(h\in\R^+\).
Denoting \(f[x_i]\) by $f_i$ , \(0\leq i\leq n\), we recursively define the operator $\nabla$ such that
\[ \begin{cases} \nabla^0 f_i=f_i \\ \nabla^{k+1}f_i=\nabla^{k}f_{i+1}-\nabla^{k}f_{i} \end{cases} \]

\begin{enumerate}
	\setcounter{enumi}{4}
	\item Show that \(\forall i,k,\in\N, f[x_i,\cdots,x_{i+k}]=\frac{1}{k!h^k}\nabla^{k}f_i\).
	\begin{proof}
	We prove by induction.
	When $k=1$, we have already proved in the previous step that
	\[ f[x_0,x_{1}]=\frac{f(x_{1})-f(x_0)}{x_1-x_0} \]
	The labeling of the points does not make a difference, and the difference between consecutive nodes is $h$, so
	\[ f[x_i,x_{i+1}]=\frac{f(x_{i+1})-f(x_i)}{h} \]
	Assume this holds for \(k=m\), which means
	\[ f[x_i,x_{i+1},\cdots,x_{i+m}]=\frac{1}{m!h^m}\nabla^mf_i \]
	According to the previous proof, for $m+1$,
	\begin{align*}
	f[x_i,x_{i+1},\cdots,x_{i+m+1}]&=\frac{f[x_{i+1},x_{i+2},\cdots,x_{i+m+1}]-f[x_{i},x_{i+2},\cdots,x{i+m}]}{x_{i+m+1}-x_{i}}\\
	&=\frac{\frac{\nabla^mf_{i+1}}{m!h^m}-\frac{\nabla^mf_{i}}{m!h^m}}{(m+1)h}\\
	&=\frac{\nabla^{m+1}f_i}{(m+1)!h^{m+1}}
	\end{align*}
	By the principle of mathematical induction, we have it true for all \(i,k\in\N\).
	\end{proof}
	\item Observing that \(\binom{s}{k}=\frac{1}{k!}\prod_{j=0}^{k-1}(s-j)\), prove that 
	\[ P^n(x)=f_0 + \sum_{k=1}^{n}\binom{s}{k}\nabla^{k} f_0, \quad \textrm{where } s=\frac{x-x_0}{h} \]
	\begin{proof}
	According to the previous proof,
	\begin{align*}
	P^n(x)&=f(x_0)+\sum_{k=1}^{n} f[x_0,\cdots, x_j] \prod_{j=0}^{k-1}(x-x_j)\\
	&=f_0+\sum_{k=1}^{n} f[x_0,\cdots, x_j] \prod_{j=0}^{k-1}(x-x_0-jh)\\
	&=f_0+\sum_{k=1}^{n} f[x_0,\cdots, x_j] h^k \prod_{j=0}^{k-1}\left(\frac{x-x_0}{h}-j\right)\\
	&=f_0+\sum_{k=1}^{n} f[x_0,\cdots, x_j] k!h^k\cdot\frac{1}{k!} \prod_{j=0}^{k-1}\left(s-j\right)=f_0+\sum_{k=1}^{n} \nabla^kf_0 \binom{s}{k}
	\end{align*}
	\end{proof}
	\item Write the pseudocode of an algorithm which takes a step $h$ as input, a number of nodes, the value of $f$ at each of those nodes, and a value $x$.
	The algorithm should return \(P^{n}(x)\approx f(x)\).
	\ifnum\webview=1
		\begin{algorithm2e}[H]
	\else
		\begin{algorithm2e}[htbp]
	\fi
		\KwIn{Step length $h$, number of nodes $n+1$, the value of $f$ at those nodes, stored in vector $y$, and a value $x$}
		\KwOut{Estimation of $f$ at $x$ using interpolation with polynomial of degree at most $n$, denoted as $f_x$}
		A=zeros(n+1,n+1)\;
		\For{i=1:n+1}
		{
			\(A(i,1)=y(i)\)\;
		}
		\(a(1)=A(1,1)\)\;
		\For{j=2:n+1}
		{
			\For{i=j:n+1}
			{
				\(A(i,j)=\frac{A(i,j-1)-A(i-1,j-1)}{(j-1)h}\)\;
			}
			\(a(j)=A(j,j)\)\;
		}
		\(f_x=0\)\;
		\For{i=0:n}
		{
			\( f_x=f_x+a(i)x^i \)
		}
		\caption{Newton's interpolation with evaluation}
	\end{algorithm2e}
\end{enumerate}