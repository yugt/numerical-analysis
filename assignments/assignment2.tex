%!TEX root = main.tex
\titleformat{\section}{\normalfont\large\bfseries}{Exercise \thesection\ --- }{1pt}{}
\chapter{Topology}

\section{Connected space}
In this exercise we provide alternative definitions for a connected space.
Let $X$ be a metric space.
We want to prove that the following conditions are equivalent
\begin{enumerate}[(i)]
	\item The only subsets that are both open and closed in $X$ are $X$ and $\varnothing$;
    \item It is impossible to write $X$ as the union of two disjoint, non-empty open subsets;
    \item It is impossible to write $X$ as the union of two disjoint non-empty closed subsets;
    \item There is no continuous, surjective application from $X$ into \(\{0, 1\}\subset\R\);
\end{enumerate}
\begin{enumerate}
	\item Prove that (i), (ii), and (iii) are equivalent.
    \begin{proof}
    The sketch of the proof is to prove (i)$\iff$(ii) and (i)$\iff$(iii), then (i), (ii), (iii) are equivalent by transitivity.
    \begin{itemize}
    	\item[(i)$\implies$(ii)]
        Assume it is possible to write $X$ as the union of two disjoint, non-empty open subsets $S_1$ and $S_2$, \textit{i.e.},
        \begin{align*}
        	&S_1\cup S_2=X,S_1\cap S_2=\varnothing \\
            &S_1\subset X, S_2\subset X\\
            &S_1\neq\varnothing, S_2\neq\varnothing
        \end{align*}
        By definition,
        \[ S_2=X\backslash S_1 \]
        Since $S_1$ is open, its complement, $S_2$ is closed.
        But by definition, $S_2$ is also open.
        Thus, we find a subset $S_2$ with \(S_2\neq X,S_2\neq\varnothing\) and is both open and closed, which causes contradiction to (i).
        \item[(ii)$\implies$(i)]
        Assume there is a subset $S$ which is both open and closed with \(S\subset X, S\neq\varnothing\).
        Since $S$ is closed, its compliment \(X\backslash S\) is open.
        By definition,
        \[ S\cup (X\backslash S)=X \quad S\cap(X\backslash S)=\varnothing \]
        Since \(X\backslash S\) is open, $S$ is also open, we have found a way to write $X$ as the union of two disjoint, non-empty open subsets, which causes contradiction to (ii).
    	\item[(i)$\implies$(iii)]
        Assume it is possible to write $X$ as the union of two disjoint, non-empty closed subsets $S_1$ and $S_2$, \textit{i.e.},
        \begin{align*}
        	&S_1\cup S_2=X,S_1\cap S_2=\varnothing \\
            &S_1\subset X, S_2\subset X\\
            &S_1\neq\varnothing, S_2\neq\varnothing
        \end{align*}
        By definition,
        \[ S_2=X\backslash S_1 \]
        Since $S_1$ is closed, its complement, $S_2$ is open.
        But by definition, $S_2$ is also closed.
        Thus, we find a subset $S_2$ with \(S_2\neq X,S_2\neq\varnothing\) and is both open and closed, which causes contradiction to (i).
        \item[(iii)$\implies$(i)]
        Assume there is a subset $S$ which is both open and closed with \(S\subset X, S\neq\varnothing\).
        Since $S$ is open, its compliment \(X\backslash S\) is closed.
        By definition,
        \[ S\cup (X\backslash S)=X \quad S\cap(X\backslash S)=\varnothing \]
        Since \(X\backslash S\) is closed, $S$ is also closed, we have found a way to write $X$ as the union of two disjoint, non-empty closed subsets, which causes contradiction to (iii).
    \end{itemize}
    From the proofs above, we can conclude (i)$\iff$(ii) and (i)$\iff$(iii), which implies (i), (ii), (iii) are equivalent.
    \end{proof}
    \item Assume (iv) is not false and show that (iii) is false.
    \begin{proof}
    We can write \(\{0,1\}\) as the union of two disjoint non-empty closed subsets
    \[ \{0,1\}=\{0\}\cup\{1\} \]
    If there is a continuous, surjective function \(f:X\to\{0,1\}\), then we have 
    \[ f^{-1}(\{0\}) \quad f^{-1}(\{1\})  \]
    are both closed in $X$.
    Since $f$ is surjective, both of \(f^{-1}(\{0\})\) and \(f^{-1}(\{1\})\) are not empty.
    And since \(\{0\}\) is complement to \(\{1\}\) in the image of $f$,
    \[ f^{-1}\left(\{0\}^c\right)=\left(f^{-1}\{0\}\right)^c\]
    Then, we can write $X$ as the union of two disjoint non-empty closed subsets
    \[ X= f^{-1}(\{0\}) \cup f^{-1}(\{1\})\]
    which cause contradiction to (iii).
    \end{proof}
    \item Assume (iii) is not false and show that (iv) is false.
    \begin{proof}
    Assume it is possible to write $X$ as the union of two disjoint, non-empty closed subsets $S_1$ and $S_2$, \textit{i.e.},
    \begin{align*}
    	&S_1\cup S_2=X,S_1\cap S_2=\varnothing \\
        &S_1\subset X, S_2\subset X\\
        &S_1\neq\varnothing, S_2\neq\varnothing
    \end{align*}
    By definition,
    \[ S_2=X\backslash S_1 \]
    Then, we can construct a continuous function \(f:X\to \{0,1\}\)
    \[ f(x)=\begin{cases}0 & x\in S_1\\ 1 & x\in S_2 \end{cases} \]
    and it is indeed surjective.
    This causes contradiction to (iv).
    \end{proof}
\end{enumerate}


\section{Intermediate value theorem}
In this exercise we prove the intermediate value theorem presented in the lecture slides.
\begin{theorem}[Intermediate Value Theorem]
Let $X$ be a connected metric space, and \(a,b \in X\).
If \(f: X \to\R\) is continuous then it takes all the values between \(f(a)\) and \(f(b)\).
\end{theorem}
\begin{enumerate}
	\item Let $X$ and $Y$ be two metric spaces, $A$ be a connected subset of $X$, and \(f: X\to Y\) be a continuous map.
    Show that \(f(A)\) is connected.\\
    \textit{Hint:} use one of the characterizations of a connected set from exercise 1.
    \begin{proof}
    Assume \(f(A)\) is not connected, then it has a subset $S$ which is both open and closed, and \(S\neq\varnothing,S\neq f(A)\).
    Then we can write \(f(A)\) as the union of two disjoint, non-empty open subsets
    \[ f(A)=S\cup(f(A)\backslash S) \]
    By the definition of continuous function,
    \begin{align*}
    	\text{open } S\subset f(A)&\implies \text{ open }f^{-1}(S)\subset A \\
    	\text{open } S^c\subset f(A)&\implies \text{ open }f^{-1}(S^c)\subset A \implies\text{ open }\left(f^{-1}(S)\right)^c\subset A 
    \end{align*}
    Then we can also write $A$ as the union of two disjoint, non-empty open subsets
    \[ A=f^{-1}(S)\cup \left(f^{-1}(S)\right)^c \]
    which leads to contradiction to the connectedness of $A$.
    \end{proof}
    \item  Let $A$ be a subset of $\R$.
    We want to prove that $A$ is connected if and only if $A$ is an interval.
    \begin{enumerate}[(a)]
    	\item Show that it is true for the empty set, and for all the subsets of $\R$ composed of a single element.
        \begin{proof}
        By definition, $\varnothing$ is both closed and open, and the only subset that is both closed and open is itself, as well as $\varnothing$.
        Then, $\varnothing$ is connected by definition.
        For those subsets which only contain a single element, there are only two different subsets: themselves and $\varnothing$, which are both closed and open.
        Thus, the subsets of $\R$ that only contain one element are also connected by definition.
        \end{proof}
        \item Assuming that $A$ is not an interval prove that $A$ is not connected.
        \begin{proof}
        The property of interval \(S\subseteq\R\) ensures that if \(x<y\in S\), and \(z\in\R\) such that \(x<z<y\), then \(z\in S\).
        If $A$ is not an interval, then \(\exists x,y\in A\) and \(z\in\R\backslash A\) such that \(x<z<y\).
        Consider the sets \(A_1=A\cap(-\infty,z)\) and \(A_2=A\cap(z,+\infty)\).
        Then $A_1$ and $A_2$ are open by definition of the subset topology on $A$.
        Neither of them is empty since \(x\in A_1\) and \(y\in A_2\).
        They are disjoint since \(A_1\cap A_2=(-\infty,z)\cap(z,+\infty)=\varnothing\).
        And their union is $A$, since \(z\notin A\).
        Then, we can write \(A_1|A_2\) us a separation of $A$.
        It follows by definition that $A$ is disconnected.
        \end{proof}
        \item We now prove the converse.
        \begin{enumerate}[i]
        	\item  Show that it suffices to prove that $\R$ is connected.
            \begin{proof}
            If a subset \(A\subset\R\) is an interval, suppose it has a open and closed subset \(T\subset A\) with
			\[ T\neq A \quad T\neq\varnothing \]
			Then there exists an element \(x\in T\) and an element \(z\in A\backslash T\).
			Without loss of generality, assume \(x<z\).
			Put \(S=T\cap[x,z]\).
			Then $S$, being the intersection of two closed sets, is closed.
			It is also bounded above, since $z$ is the upper bound.
			Since it is closed and has an upper bound, \(p:=\sup S\in S\).
			Since \(p\in[x,z],p\leq z\)
			As \(z\in A\backslash S\), \(p\neq z\) and so \(p<z\).
			Now $T$ is also an open set and \(p\in T\).
			So there exist \(a<b\in A\) such that \(p\in(a,b)\subseteq T\).
			Let $t$ be such that \(p<t<\min(b,z)\).
			So \(t\in T\) and \(t\in[p,z]\).
			Thus, \(t\in T\cap[x,z]=S\).
			This is a contradiction since \(t>p\) and $p$ is the supremum of $S$.
			Hence, our assumption is false and consequently \(T=A\) or \(T=\varnothing\).
            So the only remaining case is $\R$.
            \end{proof}
            \item Let $U$ be a subset of $\R$, different from $\R$, that is both open and closed in $\R$.
            Find a contradiction and conclude that $\R$ is connected.\\
            \textit{Hint:} observe that a closed and non-empty set having an infimum has a minimum.
            \begin{proof}
            If a proper non-empty subset \(U\subset\R\) is closed, then it has a minimum.
			At the minimum, it does not have a neighborhood, which contradicts to the openness.
			If it has a neighborhood, then there is contradiction to the definition of minimum.
			Thus, a proper non-empty subset of $\R$ cannot be both open and closed.
			Thus $\R$ has only two subsets that are both open and closed: $\varnothing$ and itself.
			It follows that $\R$ is connected by definition.
            \end{proof} 
        \end{enumerate}
    \end{enumerate}
    \item Conclude the proof of the intermediate value theorem.
    \begin{proof}
	We have already proved the connectedness of \(f(A)\).
	And we have just proved connected subsets of $\R$ are all intervals.
	Then \(f(A)\) is an interval which contains both \(f(a)\) and \(f(b)\).
	And since it is an interval, it also contains all the values between them.
    \end{proof}
\end{enumerate}


\section{Extreme value theorem}
In this exercise we prove the extreme value theorem presented in the lecture slides.
\begin{theorem}[Extreme Value Theorem]
Let $X$ be a metric space, \(f:X \to\R\) a continuous map, and $A$ be a non-empty compact subset of $X$.
Then $f$ is bounded and reaches both its upper and lower bounds.
\end{theorem}
\begin{enumerate}
	\item Let $X$ be a metric space and $A$ a subset of $X$.
    \begin{enumerate}
    	\item Prove that if $X$ is compact and $A$ is closed in $X$, then $A$ is a compact subset of $X$.
        \begin{proof}
        Since $X$ is compact, we can write it as 
        \[ X\subseteq \bigcup_{i\in I}U_i \]
        where \(U_i\) forms a finite open cover of $X$.
		Since $A$ is a closed subset of $X$, \(X\backslash A\) is an open subset of $X$, so we can take the union
        \[ X\subseteq \left(\bigcup_{i\in I}U_i \right)\cup (X\backslash A)  \]
        So there exists a finite subcovering \(U_{i_1},U_{i_2},\cdots,U_{i_k},X\backslash A\),
        \[ X\subseteq U_{i_1}\cup U_{i_2}\cup \cdots\cup U_{i_k}\cup (X\backslash A) \]
        Therefore,
        \[ A\subseteq U_{i_1}\cup U_{i_2}\cup \cdots\cup U_{i_k}\cup (X\backslash A) \]
        Since \(A\cap (X\backslash A)=\varnothing\), we can reduce it to
        \[ A\subseteq U_{i_1}\cup U_{i_2}\cup \cdots\cup U_{i_k} \]
        Hence, $A$ has a finite subcovering and thus is compact.
        \end{proof}
        \item Show that if $A$ is a compact subset of $X$ then $A$ is closed in $X$.
        \begin{proof}
        We shall show that $A$ contains all its limit points and hence is closed.
        Let \(p\in X\backslash A\), then for each \(a\in A\), there exists open sets
        \[ U_a:=B(a,d(a,p)/2) \quad V_a:=B(p,d(a,p)/2) \]
        such that \(a\in U_a\), \(p\in V_a\) and \(U_a\cap V_a=\varnothing\).
        Then
        \[ A\subseteq \bigcup_{a\in A} U_a \]
        As $A$ is compact, there exist \(a_1,a_2,\cdots,a_n\) in $A$ such that
        \[ A\subseteq U_{a_1}\cup U_{a_2}\cup\cdots\cup U_{a_n} \]
        Define
        \begin{align*}
        U&:=U_{a_1}\cup U_{a_2}\cup\cdots\cup U_{a_n}\\
        V&:=V_{a_1}\cap V_{a_2}\cap\cdots\cap V_{a_n}
        \end{align*}
        Then \(p\in V\) and \(V_a\cap U_a=\varnothing\) implies \(V\cap U=\varnothing\), which in turn implies \(V\cap A=\varnothing\).
        So $p$ is not a limit point of $A$, and $V$ is an open set containing $p$ which does not intersect $A$.
        Hence $A$ contains all of its limit points and therefore is closed.
        \end{proof}
    \end{enumerate}
    \item Prove that a subset of $\R$ is compact if and only if it is closed and bounded.
    \begin{proof}
    From the previous proof, we already have a compact subset of a metric space is closed.
    We just need it to be bounded.
    Assume a subset \(A\subseteq\R\) is unbounded, then it can be written as
    \[ A\subseteq \bigcup_{n=1}^\infty (-n,n) \]
    But \(\{(-n,n):n\in\N^*\}\) does not have any finite subcovering of $A$ since it is unbounded.
    Therefore $A$ is not compact, which leads to contradiction.
    Hence all compact subsets of $\R$ are bounded.
    Together with (a), we have all compact subsets of $\R$ are closed and bounded.
    The converse is a special case of the Heine-Borel theorem.
    \begin{theorem}[Heine-Borel Theorem]\label{Heine-Borel}
    Every closed bounded subset of $\R$ is compact.
    \end{theorem}
    \begin{proof}
    If $A$ is a closed bounded subset of $\R$, then \(A \subseteq [a,b]\), for some \(a<b\in\R\).
    Since the space \([a,b]\) is homeomorphic to \([0,1]\), we only need to prove \([0,1]\) is compact.
    Let \(O_i,i\in I\) be any open covering of \([0,1]\).
	Then \(\forall x\in[0,1],\exists O_i\) such that \(x\in O_i\).
	As \(O_i\) is open and contains $x$, there exists an interval $U_x$, open in \([0,1]\) such that \(x\in U_x\subseteq O_i\).
	Now define a subset \(S\subset[0,1]\) as follows:
	\[ S=\Set{z}{[0,z]\textrm{ can be covered by a finite number of the sets }U_x} \]
	so that
	\[ z\in S\implies [0,z]\subseteq U_{x_1}\cup U_{x_2}\cup\cdots\cup U_{x_n}\cup U_x \]
	Now let \(x\in S\) and \(y\in U_x\).
	Without loss of generality, we assume \(x\leq y\), then as $U_x$ is an interval containing $x$ and $y$, \([x,y]\subseteq U_x\).
	\[ [0,y]\subseteq U_{x_1}\cup U_{x_2}\cup\cdots\cup U_{x_n}\cup U_x \]
	and hence \(y\in S\).
	So for each \(x\in[0,1]\), \(U_x\cap S=U_x\) or \(U_x\cap S=\varnothing\).
	This implies that
	\[ S=\bigcup_{x\in S} U_x \quad\textrm{and}\quad [0,1]\backslash S=\bigcup_{x\notin S} U_x \]
	Thus we have that $S$ is open in \([0,1]\).
	But \([0,1]\) is connected.
	Therefore, \(S=[0,1]\) or \(S=\varnothing\).
	However, \(0\in S\), and so \(S=[0,1]\).
	That means \([0,1]\) can be covered by finite number of $U_x$, so
	\[ [0,1]\subseteq U_{x_1}\cup U_{x_2}\cup\cdots\cup U_{x_m} \]
	But each \(U_{x_i}\) is contained in an \(O_i,i\in I\).
	Hence,
	\[ [0,1]\subseteq O_{i_1}\cup O_{i_2}\cup\cdots\cup O_{i_m} \]
	and we have shown that \([0,1]\) is compact, which indicates all the closed intervals \([a,b]\) are compact.
    As \([a,b]\) is compact and $A$ is a closed subset, $A$ is compact.
    \end{proof}
    \end{proof}
    \item Complete the proof of the extreme value theorem.
	We apply Theorem 2.79 from the lecture slides.
	\begin{theorem}[Compactness and continuity]
	Let $X$ and $Y$ be two metric spaces and \(f:X\to Y\) be a continuous function.
	Then for any compact subset \(A\subset X\), \(f(A)\subset Y\) is compact.
	\end{theorem}
	Since $A$ is a non-empty compact subset of $X$, \(f(A)\) is compact on $\R$.
	Thus, \(f(A)\) is closed and bounded.
	Since it is bounded, it has a supremum and an infimum.
	Since it is closed, all the limit points are in \(f(A)\).
	Thus, \(\sup f(A)\in f(A)\) and \(\inf f(A)\in f(A)\).
\end{enumerate}


\section{Rolle's theorem}
Reasoning by induction and applying the extreme values theorem, prove Rolle's theorem.
\begin{theorem}[Rolle's Theorem]
\label{rolle}
Let $f$ be a \(C^n[a, b]\) function, for \(a,b\in\R\), which has $n+1$ distinct roots in \([a,b]\).
Then there exists \(c \in (a, b)\) such that \(f^{(n)}(c)=0\).
\end{theorem}
\begin{proof}
We first need Fermat's theorem.
\begin{theorem}[Fermat's Theorem]
Let $f$ be defined on \([a,b]\) and let \(c\in(a,b)\) be such that \(f(c)\geq f(x)\) for all \(x\in(a,b)\).
Then either \(f'(c)=0\) or \(f'(c)\) does not exist.
\end{theorem}
\begin{proof}
Let \(\{x_n\}\) be any sequence in \((a,b)\) with \(x_n\to c\) (and \(x_n\neq c\)) so \(f(c)\geq f(x_n)\).
Now whenever \(c>x_n\), we have
\[ \frac{f(x_n)-f(c)}{x_n-c}\geq0 \]
But whenever \(c<x_n\), we have
\[ \frac{f(x_n)-f(c)}{x_n-c}\leq0 \]
If \(f'(c)\) exists then we must have \(f'(c)=0\).
Otherwise, \(f'(c)\) does not exist.
\end{proof}
We first prove the Rolle's theorem with the \(n=1\) case.
If $f$ is constant on \([a,b]\), then its derivative is zero and so any \(c\in(a,b)\) satisfies the conclusion of the theorem.
So we assume that $f$ is not constant on \([a,b]\).
By the Extreme Value Theorem, $f$ attains an absolute maximum and an absolute minimum on \([a,b]\).
Since $f$ is not constant, at least one of these absolute extrema must occur at \(c\in(a,b)\).
Then since $f$ is differentiable on \((a,b)\), an application of Fermat's Theorem gives \(f'(c)=0\) as desired.

And Fermat's theorem can also be generalized by induction:
assume $f$ is at least $k$ times differentiable, and replace all the $f$ with \(f^{(k)}\).
Let \(c\in(a,b)\) be such that \(f^{(k)}(c)\geq f^{(k)}(x)\) for all \(x\in(a,b)\).
Then either \(f^{(k+1)}(c)=0\) or \(f^{(k+1)}(c)\) does not exist.

Assume Rolle's theorem holds for \(n=k\), then \(\exists c_1\in(a,b)\) such that \(f^{(k)}(c_1)=0\).
If \(f^{(k)}\) is constant on \([a,b]\), then its derivative is zero and so any \(c\in(a,b)\) satisfies the conclusion of the theorem.
So we assume that \(f^{(k)}\) is not constant on \([a,b]\).
By the Extreme Value Theorem, \(f^{(k)}\) attains an absolute maximum and an absolute minimum on \([a,b]\).
Since \(f^{(k)}\) is not constant, at least one of these absolute extrema must occur at \(c_2\in(a,b)\).
Then since $f$ is \(k+1\)-times differentiable on \((a,b)\), an application of Fermat's Theorem gives \(f^{(k+1)}(c_2)=0\) as desired.

According to the principle of mathematical induction, Rolle's theorem holds for all \(n\in\N^*\).
\end{proof}


\section{Continuity}
In this exercise we provide alternative characterizations for a continuous function.
Let $X$ and $Y$ be two metric spaces, $f$ be a function from $X$ into $Y$, and \(a\in X\).
We want to prove that the following conditions are equivalent
\begin{enumerate}[(i)]
	\item For all \(\varepsilon\in\R_+^*\), there exists \(\eta\in\R_+^*\) such that \(f(B(a,\eta))\subset B(f(a),\varepsilon)\);
    \item For all \(\varepsilon\in\R_+^*\), there exists \(\eta\in\R_+^*\) such that \(d(f(x),f(a))<\varepsilon\) when \(d(a,x)<\eta\), for \(x\in X\);
    \item For any neighborhood $V$ of \(f(a)\), there exists a neighborhood $U$ of $a$ such that \(f(U)\subset V\);
    \item For any neighborhood $V$ of \(f(a)\), \(f^{-1}(V)\) is a neighborhood of $a$;
\end{enumerate}
\begin{enumerate}
	\item Show that (i) and (ii) are equivalent;
    \begin{proof}
    The open ball on a metric space \((X,d)\) is defined to be
    \[ B_d(a,\eta):=\Set{x\in X}{d(x,a)<\eta} \]
    while its image is
    \[ f(B_d(a,\eta)):=\Set{f(x)\in Y}{d(x,a)<\eta} \]
    And it follows that
    \[ B_d(f(a),\varepsilon)=\Set{y\in Y}{d(y,f(a))<\varepsilon} \]
    Since \(x\in X\implies f(x)\in Y\), we can replace the $y$ with \(f(x)\),
    \[ B_d(f(a),\varepsilon)=\Set{f(x)\in Y}{d(f(x),f(a))<\varepsilon} \]
    By the definition of subset relation,
    \[ f(B(a,\eta))\subset B(f(a),\varepsilon) \implies \left(d(x,a)<\eta\implies d(y,f(a))<\varepsilon\right) \]
    The converse ((ii)$\implies$(i)) is more tricky, since if we directly take the $\eta$ in (ii) as the $\eta$ in (i), we will get a $\subseteq$ relation according to definition.
    Thus, we need to take \(\eta_1=\eta_2/2\), where $\eta_1$ is the $\eta$ in (i) and $\eta_2$ is the $\eta$ in (ii).
    Then
    \[ \left(d(x,a)<\eta\implies d(y,f(a))<\varepsilon\right) \implies f(B(a,\eta/2))\subset f(B(a,\eta)) \subseteq B(f(a),\varepsilon) \]
    \end{proof}
    \item Consider a neighborhood of \(f(a)\) and prove that (i) implies (iii).
    \begin{proof}
    This is trivial by taking \(U=B(a,\eta)\) and \(V=B(f(a),\varepsilon)\).
    \end{proof}
    \item Observe that if $V$ is a neighborhood of \(a \in X\), then any subset of $X$ containing $V$ is a neighborhood of $a$.
    Conclude that (iii) implies (iv).
    \begin{proof}
    By the definition of subset relation,
    \[ V\subseteq W\implies \forall a\in V,a\in W \]
    This shows that $W$ is also a neighborhood of $a$.
    By writing $U$ as \(f^{-1}(f(U))\), we have
    \[ U=f^{-1}(f(U))\subset f^{-1}(V) \]
    Since $U$ is a neighborhood of $a$, \(f^{-1}(V)\) is also a neighborhood of $a$.
    \end{proof}
    \item Consider \(V = B(f(a), \varepsilon)\), for some $\varepsilon$, and prove that (iv) implies (i).
    \begin{proof}
    An open set $S$ on a metric space \((X,d)\) has the property that 
    \[ \forall x\in S, \exists B(x,\eta)\subset S \]
    By (iv), \(f^{-1}(V)=f^{-1}(B(f(a), \varepsilon))\) is a neighborhood of $a$, which is an open set, and $X$ is a metric space, we have already found an $\eta$ for (i) to hold.
    \end{proof}
\end{enumerate}