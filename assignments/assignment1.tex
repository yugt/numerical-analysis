%!TEX root = ../main.tex
\chapter{Introduction}

\section{Metric space}
Let $X$ be a metric space
\begin{enumerate}
	\item Show that $\varnothing$ and $X$ are closed in $X$.
    \begin{proof}
    According to the definition of openness, a subset $U$ of $X$ is \emph{open} if 
    \[\forall x\in U, \exists \varepsilon>0 \textrm{ such that } (y\in X, d(x,y)<\varepsilon)\implies y\in U \]
    And the definition of closed set is a complement of an open set.
    Since $\varnothing$ does not have any elements, all statements with \(x\in\varnothing\) is trivially true.
    Thus, $\varnothing$ is open, which implies its complement
    \[X\backslash\varnothing=X\]
    is closed.
    Then, we prove $X$ is also open.
    According to the definition,
    \[\forall x\in X, \exists \varepsilon>0 \textrm{ such that } (y\in X, d(x,y)<\varepsilon)\implies y\in X \]
    This is also trivially true since we have \(y\in X\implies y\in X\).
    So $X$ is open, and its complement
    \[X\backslash X=\varnothing\]
    is closed.
    \end{proof}
    \item Prove that the intersection of a collection of closed subsets of $X$ is closed in $X$.
    \begin{proof}
    By theorem, a metric space is a topological space, so we can directly apply the property that 
    \begin{quote}
    	Any union, finite or not, of open subsets of $X$ is open.
    \end{quote}
    Then we take the complement and apply de Morgan's law to get:
    any intersection, finite or not, of closed subsets of $X$ is closed.
    \end{proof}
    \item Prove that the union of a finite collection of closed subsets of $X$ is closed in $X$.
    \begin{proof}
    Similarly, we quote another property of topological spaces
   \begin{quote}
    	Any intersection of finitely many open subsets of $X$ is open.
    \end{quote}
    Then we take the complement and apply de Morgan's law to get:
   the union of finitely many closed subsets of $X$ is closed.
    \end{proof}
\end{enumerate}


\section{Continuity}
For each of the following question prove your result.
\begin{enumerate}
	\item Give an example of function which is continuous but not uniformly continuous;
    \paragraph{Example:} \(f(x)=x^2\).
    \begin{proof}
    Continuity can be proved by choosing \(\delta=\frac{\varepsilon}{1+|x+x_0|}\) in the \(\varepsilon-\delta\) definition.
    \begin{align*}
    	&\forall\varepsilon>0,\exists\delta=\frac{\varepsilon}{1+|x+x_0|}>0\textrm{ such that } |x-x_0|<\delta\\
        &\implies |f(x)-f(x_0)|=|x-x_0|\cdot|x+x_0|<\frac{\varepsilon}{1+|x+x_0|}\cdot|x+x_0|<\varepsilon
    \end{align*}
	And the property of uniform continuous can be violated by choosing \(x=\frac{\varepsilon}{\delta},y=x+\frac{\delta}{2}\), which produces
	\[ |x-y|=\frac{\delta}{2}<\delta \textrm{ while } |f(y)-f(x)|=\left|x\delta+\frac{\delta^2}{4}\right|=\left|\varepsilon+\frac{\delta^2}{4}\right|>\varepsilon \]
	So this is a continuous but not a uniformly continuous function.
    \end{proof}
	\item Give an example of function which is uniformly continuous but not Lipschitz continuous.
	\paragraph{Example:} \(f(x)=\sqrt{x}\) on its domain $\R^+$.
	\begin{proof}
    Uniform continuity can be proved by choosing \(\delta=\varepsilon^2\) in the \(\varepsilon-\delta\) definition.
    \begin{align*}
    	&\forall\varepsilon>0,\exists\delta=\varepsilon^2>0\textrm{ such that } |x-y|<\delta\\
        &\implies |f(x)-f(y)|=|\sqrt{x}-\sqrt{y}|\\
        &\leq \sqrt{|\sqrt{x}-\sqrt{y}|\cdot|\sqrt{x}+\sqrt{y}|}\\
        &=\sqrt{|x-y|}=\sqrt{\varepsilon^2}=\varepsilon
    \end{align*}
	And the property of Lipschitz continuous can be violated by writing
	\[ |f(x)-f(y)|=|\sqrt{x}-\sqrt{y}|=\frac{1}{\sqrt{x}+\sqrt{y}}|x-y| \]
	By choosing \(y=0\), we need the Lipschitz constant to be the supreme 
	\[ K\geq \sup_{x>0}\frac{1}{\sqrt{x}} \]
	However, 
	\[ \lim_{x\to0^+}=+\infty \]
	which means we can't find such a Lipschitz constant.
	So this is a uniformly continuous but not a Lipschitz continuous function.
    \end{proof}
\end{enumerate}


\section{Cardinality}
\begin{enumerate}
	\item Prove that $\N$, $\Z$, and $\Q$ have the same number of elements.
	\begin{proof}
	We can construct a bijective function from $\N$ to $\Z$:
	\[f(n)=(-1)^n\lfloor \frac{n+1}{2}\rfloor\quad n\in\N \]
	Injectivity is trivial because this function only returns an integer.
	Surjectivity is proved by
	\[n=\begin{cases} 4k &\to 2k \\ 4k+1 &\to -(2k+1) \\ 4k+2 &\to 2k+1 \\ 4k+3 &\to -2(k+1) \end{cases} \]
	which covers the positive even numbers, negative odd numbers, positive add numbers, negative even numbers and zero, whose union is $\Z$ and are disjoint.
	Thus, the mapping is both injective and surjective, which is bijective.
	Then, we can construct a bijective mapping \(f:\N^*\times\N^*\to\Q^+\)
	\begin{center}
	\begin{tabular}{ccccc}
	1/1 & 1/2 & 1/3 & 1/4 & $\cdots$\\
	2/1 & 2/2 & 2/3 & 2/4 & $\cdots$\\
	3/1 & 3/2 & 3/3 & 3/4 & $\cdots$\\
	4/1 & 4/2 & 4/3 & 4/4 & $\cdots$\\
	$\vdots$ & $\vdots$ & $\vdots$ & $\vdots$ & $\vdots$
	\end{tabular}
	\end{center}
	This mapping is trivially injective since it returns a fraction of two natural numbers.
	It is also surjective because every positive rational number can be written as a ratio of two integers.
	Then we can count diagonally to construct the bijective mapping \(g:\N^*\to\Q^+\).
	And the negative integers can use the same mapping with negative signs on the range, so that all the negative rational numbers are mapped.
	Finally, we map the zero in $\Z$ to the zero in $\Q$ to construct the whole map, which gets the surjectivity.
	Thus, this map is also bijective.
	The first mapping concludes \(\mathrm{Card}(\N)=\mathrm{Card}(\Z)\), and the second mapping concludes \(\mathrm{Card}(\Z)=\mathrm{Card}(\Q)\).
	The combined result is
	\[\mathrm{Card}(\N)=\mathrm{Card}(\Z)=\mathrm{Card}(\Q)\]
	\end{proof}
    \item Prove that $\R$ has more elements than $\N$.
	\begin{proof}
	We prove that the subset \([0,1]\subset\R\) has more elements than $\N$.
	Suppose we can construct a bijective map from $\N$ to \([0,1]\), then
	\begin{align*}
	n_1 &\to 0.a_{11}a_{12}a_{13}a_{14}\cdots\\
	n_2 &\to 0.a_{21}a_{22}a_{23}a_{24}\cdots\\
	n_3 &\to 0.a_{31}a_{32}a_{33}a_{34}\cdots\\
	&\vdots
	\end{align*}
	where \(a_{ij}\in\{0,1,2,3,4,5,6,7,8,9\}\).
	Then we can always find an element from \([0,1]\) that is not mapped to.
	Take \(b_1\neq a_{11},b_2\neq a_{22},\cdots,b_i\neq a_{ii},\cdots\), \(0.b_1 b_2 \cdots b_i\cdots\in[0,1]\) differs from all the elements in the range of the mapping, so the mapping is not surjective.
	This contradicts to the bijection assumption.
	Thus the bijection does not exist, and \([0,1]\) has more elements than $\N$, which implies $\R$ has more elements than $\N$.
	\end{proof}
    \item Prove that \([0, 1]\) has as many elements as $\R$.
	\begin{proof}
	We find a bijective mapping \(f:\R\to(0,1)\):
	\[f(x)=\frac{1}{2}+\frac{1}{\pi}\arctan(x) \]
	This mapping is monotone so that its inverse exists, which is clearly bijective.
	This implies \((0,1)\) has as many elements as $\R$, and adding two more elements \(\{0,1\}\) to infinite set does not change the cardinality.
	Thus,
	\[\mathrm{Card}([0,1])=\mathrm{Card}(\R)\]
	\end{proof}
\end{enumerate}


\section{Cauchy-Schwartz inequality}
% \section{Slides}
% \begin{enumerate}
	% \item 
	In the lecture the Cauchy-Schwartz inequality was proven for real numbers, prove it for the complex numbers.
	\begin{theorem}[Cauchy-Schwartz Inequality]
	\label{cauchy-schwartz}
	Let $V$ be an inner product space over \(\C\) and \(u,v\in V\). Then
	\[ |\langle u,v\rangle| \leq ||u|| \cdot ||v||. \]
	This equality occurs if and only if one of \(u,v\) is a scalar multiple of the other.
	\end{theorem}
	\begin{proof}
	If \(v=0\), both sides are zeros, so the equality is trivially true.
	If \(v\neq0\), we define
	\[\lambda=\frac{\langle u,v\rangle}{||v||^2}  \]
	Then,
	\begin{align*}
	0&\leq ||u-\lambda v||^2\\
	&=\langle u,u \rangle-\langle u,\lambda v\rangle-\langle\lambda v,u \rangle+\langle \lambda v,\lambda v\rangle\\
	&=\langle u,u \rangle-\lambda\bar{\langle u,v \rangle}-\bar\lambda\langle u,v \rangle+\lambda\bar\lambda \langle v,v \rangle\\
	&=||u||^2-\frac{|\langle u,v \rangle|^2}{||v||^2}-\frac{|\langle u,v \rangle|^2}{||v||^2}+\frac{|\langle u,v \rangle|^2}{||v||^2}\\
	&=||u||^2-\frac{|\langle u,v \rangle|^2}{||v||^2}\\
	&\implies ||u||^2\geq\frac{|\langle u,v \rangle|^2}{||v||^2} 
	\end{align*}
	Since all the squares are nonnegative, we get
	\[||u||||v||\geq |\langle u,v \rangle|^2 \]
	\end{proof}
	% \item Show that a distance is always positive.
	% \begin{proof}
	% This is directly deduced from the definition:
	% \begin{quote}
	% 	A distance on $X$ is a function \(d: X\times X\to\R^+\) such that ...
	% \end{quote}
	% which already indicates that the distance is always positive.
	% \end{proof}
% \end{enumerate}


\section{Linear algebra}
\begin{enumerate}
	\item Let $f$ be a linear map from a vector space $V_1$ into a vector space $V_2$.
    Show that the dimension of $V_1$ is the sum of the dimensions of the kernel and of the image of $f$.
	This result is called the rank nullity theorem.
	\begin{proof}
	Suppose \(\{{u}_{1},\ldots ,{u}_{m}\}\) forms a basis of \(\ker f\).
	We can extend this to form a basis of $V_1$: \(\{{u}_{1},\ldots ,{u}_{m},{w}_{1},\ldots,{w}_{n}\}\).
	Since the dimension of $\ker f$ is $m$ and the dimension of $V_1$ is \(m + n\), it suffices to show that the dimension of image of $f$ ($V_2$) is $n$.
	We can show that \(\{f({w}_{1}),\ldots,f({w}_{n})\}\) forms a basis form a basis of $V_2$.
	There exist unique scalars such that
	\begin{align*}
		v&=a_1u_1+\cdots a_m u_m+b_1 w_1+\cdots+ b_n w_n\\
		&\implies f(v)=a_1 f(u_1)+\cdots+ a_m f(u_m)+b_1 f(w_1)+\cdots+ b_n f(w_n)\\
		&\implies f(v)=b_1 f(w_1)+\cdots+ b_n f(w_n)
	\end{align*}
	Thus, \(\{f(w_1),\cdots, f(w_n)\} \) spans $V_2$.
	Then, we have to show these vectors are linearly independent to ba a basis of $V_2$.
	\begin{align*}
		c_1 f(w_1)+\cdots+c_n f(w_n)=0 \iff f(c_1w_1+\cdots+c_n w_n)=0\\
		\implies c_1w_1+\cdots+c_n w_n\in \ker f
	\end{align*}
	Then since $u_i$ span $\ker f$, there exists a set of scalars $d_i$ such that
	\[ c_1w_1+\cdots+c_n w_n=d_1u_1+\cdots+d_m u_m \]
	But since \(\{{u}_{1},\ldots ,{u}_{m}, {w}_{1}, \ldots,{w}_{n}\}\) form a basis of $V_1$, all $c_i,d_i$ must be zeros.
	Therefore, \sloppy \(\{f({w}_{1}), \ldots, f({w}_{n})\}\) is linearly independent and forms a basis of $V_2$.
	This proves that the dimension of $V_2$ is $n$ as desired.
	\end{proof}
    \item Prove that the composition of two linear maps is a linear map.
	\begin{proof}
	Suppose the two linear maps are $f$ and $g$, and a scalar \(a\in\mathbb{K}\)
	\begin{align*}
		f(x+ay)&=f(x)+a f(y)\\
		g(f(x+ay))&=g(f(x)+a f(y))=g(f(x))+a g(f(y))
	\end{align*}
	This is exactly the linearity of the composition $g(f)$.
	\end{proof}
    \item Prove that the inverse of a linear map is a linear map.
	\begin{proof}
	Suppose the linear map is $f$ and its inverse is $f^{-1}$, and a scalar \(a\in\mathbb{K}\).
	\begin{align*}
	f(x+ay)&=f(x)+a f(y)\\
	f^{-1}(f(x+ay))&=f^{-1}(f(x)+a f(y))\\
	x+ay&=f^{-1}(f(x)+a f(y)
	\end{align*}
	But we can write $x$ as \(f^{-1}(f(x))\), and $y$ as \(f^{-1}(f(y))\).
	\[f^{-1}(f(x))+af^{-1}(f(y))=f^{-1}(f(x)+a f(y)) \]
	This is exactly the linearity of $f^{-1}$.
	\end{proof}
    \item Consider \(\mathcal{C}^\infty(\R)\), the set of the functions that are infinitely differentiable over $\R$. We equip \(\mathcal{C}^\infty(\R)\) of the supremum norm.
    \begin{enumerate}[(a)]
    	\item Show that \(f(x) = \sin(nx)/n \in \mathcal{C}^\infty(\R)\).
		\begin{proof}
		We find the first four differentials
		\begin{align*}
		\mathrm{d}f&=\cos(nx)\\
		\mathrm{d}^2f&=-n\sin(nx)\\
		\mathrm{d}^3f&=-n^2\cos(nx)\\
		\mathrm{d}^4f&=n^3\sin(nx)\\
		\end{align*}
		Thus we get the recursive relation between the differentials.
		\[\mathrm{d}^{k+4}f=n^4\mathrm{d}^{k}f \]
		Since the first four are all differentiable, all differentials are differentiable.
		\end{proof}
        \item Determine the differential of $f$.
		\begin{proof}[Answer]
		According to the recursive relation,
		\begin{align*}
		\mathrm{d}^{4k+1}f&=n^{4k}\cos(nx) \\
		\mathrm{d}^{4k+2}f&=-n^{4k+1}\sin(nx)\\
		\mathrm{d}^{4k+3}f&=-n^{4k+2}\cos(nx)\\
		\mathrm{d}^{4k+4}f&=n^{4k+3}\sin(nx)\\
		\end{align*}
		\end{proof}
        \item Prove that differentiation, viewed as a linear map, is not continuous.
		\begin{proof}
		Take $x=0$ and $y=\frac{\pi}{n}$, when $n\to\infty$, $|f(x)-f(y)|=2n$ which goes to infinity.
		This contradicts to the continuity.
		\end{proof}
        \item How does this relate to theorem 1.28?
        \begin{proof}[Answer]
		This implies the differentiation does not have a finite norm.
        \end{proof}
    \end{enumerate}
\end{enumerate}

\section{Pi}
\begin{enumerate}
	\item Write the pseudocode for
    \begin{enumerate}
		\item The approximation of $\pi$ using polygons;
        \begin{proof}
        \[\pi=\lim_{n\to\infty}\frac{n}{2}\sin\frac{2\pi}{n} \]
        \end{proof}
        \item The approximation of $\pi$ using Machin's formula \(\frac{\pi}{4}=4\arctan\frac{1}{5}-\arctan\frac{1}{239}\) and Taylor series;
		\begin{proof}
			\[\pi=4\sum_{k=0}^{\infty}\frac{(-1)^k}{2k+1}\left[4\cdot5^{-(2k+1)}-239^{-(2k+1)} \right] \]
		\end{proof}
	\end{enumerate}
    \item Implement the two previous algorithms in MATLAB.
	\begin{lstlisting}[style=Matlab-editor]
    function [ p ] = PiApproximation1( n )
    	p=4;
    	t=(sqrt(2)-1); %t0
        a=t^2; %a0
    	t=t/(1+sqrt(1+t^2)); %t1
    	p=p-4*a; %p1
    	for i=1:n
        	a=2*t^3/(1-t^2);
        	t=t/(1+sqrt(1+t^2));
        	p=p-4*a*2^i;
    	end
	end

	function [ p ] = PiApproximation2( n )
		p=0;
		for k=0:n
			p=p+(-1)^k/(2*k+1)
            *(4*5^(-2*k-1)-239^(-2*k-1));
		end
		p=p*4;
	end
	\end{lstlisting}
\end{enumerate}