%!TEX root = main.tex
\renewcommand{\chaptername}{Project}
\renewcommand{\thesection}{\arabic{section}}
\titleformat{\section}{\normalfont\large\bfseries}{\thesection}{1em}{}
\setcounter{chapter}{0}


\chapter{Linear Prediction of Speech}
\begin{center}
Guangting Yu, University of Michigan - Shanghai Jiao Tong University Joint Institute, Minhang, Shanghai, 200135, China
\end{center}


\section*{Abstract}
This paper talks about numerical methods.



\section{Background}
Speech production is the result of an excitation signal generated by the contraction of the lungs when they expel air.\cite{dutoit}
It is then modified by resonances when passing through the trachea, the vocal cords, the mouth cavity, as well as various muscles.\cite{tam59}
The excitation signal is either created by the opening and closing of the vocal cords, or by a continuous flow of air.\cite{gtm181}
Introduced in the early 1960s by Fant, \textit{the source-filter} model assumes that the glottis and vocal tract are fully uncoupled.\cite{corless}
This initial idea was reused to develop the \textit{Linear Predictive} (LP) model for speech production.\cite{tam39}

In this model the speech signal is the output \(y[n]\) of an \textit{all-pole filter}\footnote{A filer whose frequency response function goes infinite at specific frequencies} \(1/A(z)\) excited by \(x[n]\).\cite{golan}
Calling \(Y(z)\) and \(X(z)\) the Z-transform of the speech and excitation, respectively, the model is described by\cite{utm}
\[ Y(z)=\frac{X(z)}{1-\sum_{i=0}^p a_i z^{-i}}=\frac{X(z)}{A_p} \]
Applying the inverse Z-transform to this equation we observe that the speech can be linearly predicted from the previous $p$ samples and some excitation:
\[ y[n] = x[n]+\sum_{i=1}^p a_i y[n-i] \]
Our goal is to explain as much as possible of \(y[n]\) through the $a_i$ , \textit{i.e.}, we look at \(x[n]\) as an error, and we strive at rendering it as small and simple as possible.\cite{gtm135}
For the sake of clarity we therefore rename \(x[n]\) into \(e[n]\).\cite{cc12}
The question we want to answer is how to select the $a_i$ such as to minimize the energy
\[ E=\sum_{m=-\infty}^\infty e^2[m] \]

\begin{enumerate}
	\item Show that
	\[ \sum_{m=-\infty}^{\infty} y[m]y[m-i]=\sum_{i=1}^p a_i\sum_{m=-\infty}^{\infty} y[m-i]y[m-i], \quad i=1,\cdots,p  \]
	Since those sums are infinite they cannot be computed, and as such need to be truncated.\cite{karris}
	This can be achieved by applying the covariance method, which consists in windowing the error
	\[ E_n=\sum_{m=n}^{n+N-1}\left(y[m]-\sum_{i=1}^p a_i y[m-k] \right)^2 \]
	\item Prove that
	\begin{align*}
	\phi_n(k,0)&=\sum_{i=0}^p a_i\phi_n(k,i) \\
	\phi_n(k,i)&=\sum_{m=n}^{n+N-1} y[m-k]y[m-k]
	\end{align*}
	\item Conclude that
	\[ \begin{pmatrix} \phi(1,0)\\ \vdots \\ \phi(p,0) \end{pmatrix}=\begin{pmatrix} \phi(1,1) & \cdots & \phi(1,p) \\ \vdots & & \vdots \\ \phi(p,1) & \cdots & \phi(p,p) \end{pmatrix}\begin{pmatrix} a_1 \\ \vdots \\ a_p \end{pmatrix} \]
	Determining the optimal value for the \(a_i, 1\leq i\leq p\), implies inverting the matrix $\Phi$.
	This can be achieved through Cholesky decomposition.\cite{gtm216,gtm135}
\end{enumerate}



\section{Linear algebra}

\subsection{Remainders}
