%!TEX root = main.tex
\titleformat{\section}{\normalfont\large\bfseries}{\thesection}{1em}{}
\renewcommand{\chaptername}{Project}
\renewcommand{\thesection}{\arabic{section}}


\chapter{Linear Prediction of Speech}
\begin{center}
Guangting Yu, University of Michigan - Shanghai Jiao Tong University Joint Institute, Minhang, Shanghai, 200135, China
\end{center}


\section*{Abstract}
This paper talks about numerical methods.


\section{Background}
Speech production is the result of an excitation signal generated by the contraction of the lungs when they expel air.\cite{dutoit}
It is then modified by resonances when passing through the trachea, the vocal cords, the mouth cavity, as well as various muscles.\cite{tam59}
The excitation signal is either created by the opening and closing of the vocal cords, or by a continuous flow of air.\cite{gtm181}
Introduced in the early 1960s by Fant, \textit{the source-filter} model assumes that the glottis and vocal tract are fully uncoupled.\cite{corless}
This initial idea was reused to develop the \textit{Linear Predictive} (LP) model for speech production.\cite{tam39}

In this model the speech signal is the output \(y[n]\) of an \textit{all-pole filter}\footnote{A filter whose frequency response function goes infinite at specific frequencies} \(1/A(z)\) excited by \(x[n]\).\cite{golan}
Calling \(Y(z)\) and \(X(z)\) the Z-transform of the speech and excitation, respectively, the model is described by\cite{utm}
\begin{equation}
Y(z)=\frac{X(z)}{1-\sum_{i=0}^p a_i z^{-i}}=\frac{X(z)}{A_p}.
\end{equation}

Applying the inverse Z-transform to this equation we observe that the speech can be linearly predicted from the previous $p$ samples and some excitation:
\begin{equation}
y[n] = x[n]+\sum_{i=1}^p a_i y[n-i].
\end{equation}


Our goal is to explain as much as possible of \(y[n]\) through the $a_i$ , \textit{i.e.}, we look at \(x[n]\) as an error, and we strive at rendering it as small and simple as possible.\cite{gtm135}
For the sake of clarity we therefore rename \(x[n]\) into \(e[n]\).\cite{cc12}
The question we want to answer is how to select the $a_i$ such as to minimize the energy
\begin{equation}
E=\sum_{m=-\infty}^\infty e^2[m].
\end{equation}




\section{Linear algebra}



\begin{definition}[Hermitian matrix]
A Matrix \(A\in\C^{n\times n}\) is \emph{Hermitian} if it is equal to the conjugate transpose of itself, \textit{i.e.},
\begin{equation*}
A=\bar{A}^T.
\end{equation*}
\end{definition}


\begin{definition}[Eigenvalues ad eigenvectors]
A number \(\lambda\in\C\) is called an \emph{eigenvalue} of \(A\in\C^{n\times n}\) if there exists \(v\in\C^n\) such that \(v\neq0\) and
\begin{equation*}
Av=\lambda v.
\end{equation*}
Such a vector $v$ is called an \emph{eigenvector}.
\end{definition}


\begin{definition}[Spectrum]
The \emph{spectrum} of \(A\in\C^{n\times n}\), denoted as \(\sigma_A\), is the set of all the eigenvalues of $A$, \textit{i.e.},
\begin{equation*}
\sigma_A\coloneqq\Set{\lambda}{\lambda \text{ is a eigenvalue of }A}
\end{equation*}
\end{definition}


\begin{definition}[Spectrum radius]
The \emph{spectrum radius} of \(A\in\C^{n\times n}\), denoted as \(\rho(A)\), is the largest absolute value of all its eigenvalues, \textit{i.e.},
\begin{equation*}
\rho(A)\coloneqq\max_{\lambda\in\sigma_A}|\lambda|.
\end{equation*}
\end{definition}


\begin{theorem}
If a matrix \(A\in\C^{n\times n}\) is Hermitian, then \(\sigma_A\subset\R\).
\end{theorem}
\begin{proof}
Take any eigenvalue \(\lambda\in\sigma_A\) and its corresponding eigenvector \(v\in\C^n\).
By definition,
\begin{equation}
Av=\lambda v.
\end{equation}
Multiply both sides by the conjugate transpose of $v$,
\begin{equation}\label{hermitianproof}
\bar{v}^T Av=\lambda \bar{v}^T v.
\end{equation}
The left hand side of the equation has the conjugate transpose
\begin{equation}
\overline{\bar{v}^T Av}^T=\bar{v}^T\bar{A}^T v.
\end{equation}
Since $A$ is Hermitian, we plug in \(\bar{A}^T=A\) on the right hand side and
\begin{equation}
\overline{\bar{v}^T Av}^T=\bar{v}^TA v,
\end{equation}
which implies that \(\bar{v}^T Av\) is also Hermitian.
Furthermore, \(\bar{v}^T Av\in\C^{1\times1}\) implies that it is a real number.
Then, according to equation \ref{hermitianproof}
\begin{equation}
\lambda=\frac{\bar{v}^T Av}{\bar{v}^T v}.
\end{equation}
The denominator is the inner product of $v$ with itself defined on \(\C^n\), so \(\bar{v}^T v\in\R_*^+\) since \(v\neq0\).
Both the numerator and the denominator are real, so the quotient $\lambda$ is also real.
Since $\lambda$ is arbitrarily selected from \(\sigma_A\), all elements in \(\sigma_A\) are real.
Furthermore, \(\sigma_A\) has at most $n$ elements, so it is a proper subset of 
\end{proof}

\begin{theorem}
If \(A,B\in\C^{n\times n}\), then \(\sigma_{AB}=\sigma_{BA}\).
\end{theorem}
\begin{proof}
Select \(\lambda\in\sigma_{AB}\).
\item If \(\lambda=0\), then there exists \(v\in\C^n,v\neq0\) such that \(ABv=\lambda v=0\).
So the matrix $AB$ is not full rank, and thus \(\det(AB)=0\).
Then \(\det(BA)=\det(AB)=0\), which implies $BA$ is not full rank either.
Thus, there exists a nonzero vector \(u\in\C^n\) such that \(BAu=0\).
This means 0 is also an eigenvalue of $BA$.
\item If \(\lambda\neq0\), then there exists \(v\in\C^n,v\neq0\) such that \(ABv=\lambda v\).
Then, multiply both sides by $B$,
\begin{equation}
BABv=\lambda Bv
\end{equation}
This also means \(BA\) has $\lambda$ as its eigenvalue with corresponding eigenvector \(Bv\).
Combine both cases, \(\lambda\in\sigma_{BA}\).
Since $\lambda$ is arbitrarily selected from \(\sigma_{AB}\), we have \(\sigma_{AB}\subseteq\sigma_{BA}\).
Exchange the name of $A$ and $B$, we also have \(\sigma_{AB}\supseteq\sigma_{BA}\).
Therefore, \(\sigma_{AB}=\sigma_{BA}\).
This directly implies \(\rho(AB)=\rho(BA)\).
\end{proof}

\begin{definition}[Norm]
Given a vector space $V$ over a subfield \(\mathbb{F}\subseteq\C\), a \emph{norm} on $V$ is a function \(||\cdot||:V\to\R\) with the following properties:
\item For all \(c\in\mathbb{F}\) and all \(u,v\in V\),
\begin{enumerate}
	\item \(||v||\geq0\);
	\item \(||cv||=|c|\cdot||v||\);
	\item \(||u+v||\leq||u||+||v||\);
	\item \(||v||=0\implies v=0\).
\end{enumerate}
\end{definition}

\begin{theorem}
The set of functions on \(A\in\C^{n\times n}\) with \(a_{ij}\in A,1\leq i,j\leq n\),
\begin{equation}\label{matrixnorm}
N_p(A)\coloneqq
\begin{cases} 
\left(\sum_{1\leq i,j\leq n}|a_{ij}|^p\right)^{\frac{1}{p}} & p\in\N^*\\
\max_{1\leq i,j\leq n}|a_{ij}| & p=+\infty
\end{cases}
\end{equation}
define a set of matrix norms.
\end{theorem}
\begin{proof}
We separate the cases where \(p=+\infty\) and \(p\in\N^*\).
\begin{itemize}
	\item For \(p=+\infty\):
	\begin{enumerate}%[Property 1]
		\item Trivially,
		\[N_\infty(A)=\max_{1\leq i,j\leq n}|a_{ij}|\ge0. \]
		\item Obviously,
		\[ N_\infty(cA)=\max_{1\leq i,j\leq n}|c\cdot a_{ij}|=\max_{1\leq i,j\leq n}|c|\cdot|a_{ij}|=|c|\max_{1\leq i,j\leq n}|a_{ij}|=|c|\cdot N_\infty(A). \]
		\item Clearly,
		\begin{align*}
		N_\infty(A+B)&=\max_{1\leq i,j\leq n}|a_{ij}+b_{ij}|\leq\max_{1\leq i,j\leq n}(|a_{ij}|+|b_{ij}|)\\
		&\leq\max_{1\leq i,j\leq n}|a_{ij}|+\max_{1\leq i,j\leq n}|b_{ij}|=N_\infty(A)+N_\infty(B).
		\end{align*}
		\item Apparently,
		\[N_\infty(A)=0\implies\max_{1\leq i,j\leq n}|a_{ij}|=0\implies \forall i,j\in[1,n]\cap\N,|a_{ij}|=0\implies A=0. \]
	\end{enumerate}
	\item For \(p\in\N^*\), only property 1,2 and 4 of the definition will be proved.
	The property 3 is called the triangle inequality and will be proved in the following context by three famous theorems: Young's inequality \ref{young}, H\"{o}lder's inequality \ref{holder} and Minkowski's inequality \ref{minkowski}.
	\begin{enumerate}
		\item Trivially,
		\[ N_p(A)=\left(\sum_{1\leq i,j\leq n}|a_{ij}|^p\right)^{\frac{1}{p}}\ge0. \]
		\item Obviously,
		\[ N_p(cA)=\left(\sum_{1\leq i,j\leq n}|c\cdot a_{ij}|^p\right)^{\frac{1}{p}}=\left(|c|^p\cdot\sum_{1\leq i,j\leq n}|a_{ij}|^p\right)^{\frac{1}{p}}=|c|\cdot N_p(A). \]
		\item We need to prove
		\[ N_p(A+B)\leq N_p(A)+N_p(B), \]
		which can be explicitly expressed as
		\[ \left(\sum_{1\leq i,j\leq n}|a_{ij}+b_{ij}|^p\right)^{\frac{1}{p}}\leq\left(\sum_{1\leq i,j\leq n}|a_{ij}|^p\right)^{\frac{1}{p}}+\left(\sum_{1\leq i,j\leq n}|b_{ij}|^p\right)^{\frac{1}{p}}. \]
		The dimension of the matrix does not matter here, so that we can compress it to sequences,
		\[ \left[\sum_{i=1}^n|a_i+b_i|^p\right]^\frac{1}{p}\leq\left(\sum_{i=1}^n |a_i|^p\right)^\frac{1}{p}+\left(\sum_{i=1}^n |b_i|^p\right)^\frac{1}{p}. \]
		This is the discrete sum case of Minkowski's inequality , which will be proved in the following context, with the support of H\"{o}lder's inequality.
		\item Apparently,
		\[N_p(A)=0\implies\sum_{1\leq i,j\leq n}|a_{ij}|^p=0\implies \forall i,j\in[1,n]\cap\N,|a_{ij}|=0\implies A=0. \]
	\end{enumerate}
\end{itemize}
\end{proof}




\begin{theorem}[Young's inequality]\label{young}
Let \(p,q\in\R_*^+\) be strictly positive real numbers such that \(\frac{1}{p}+\frac{1}{q}=1\).
Then, for any \(a,b\in\R^+\),
\[ ab\leq\frac{a^p}{p}+\frac{b^q}{q}. \]
Equality occurs if and only if
\[ b=a^{p-1}. \]
\end{theorem}
\begin{proof}
We write the left hand side of the inequality as
\[ ab=\exp\left(\frac{1}{p}\ln a^p+\frac{1}{q}\ln b^q\right). \]
Since the exponential function is strictly positive, we apply Jensen's inequality,
\[ \exp\left(\frac{1}{p}\ln a^p+\frac{1}{q}\ln b^q\right)\leq\frac{1}{p}\exp\left(\ln a^p\right)+\frac{1}{q}\exp\left(\ln b^q\right). \]
Apply the substitbution and simplification, we get
\[ ab\leq\frac{a^p}{p}+\frac{b^q}{q}. \]
\end{proof}




\begin{theorem}[H\"{o}lder's inequality]\label{holder}
Let \(p,q\in(1,+\infty)\) be strictly positive real numbers such that \(\frac{1}{p}+\frac{1}{q}=1\).
Two sequences of positive real numbers \(a_i,b_i\in\R^+,\forall i\in[1,n]\cap\N\) satisfies
\[ \sum_{i=1}^n a_ib_i\leq\left(\sum_{i=1}^n a_i^p\right)^{\frac{1}{p}}\left(\sum_{i=1}^n b_i^q\right)^{\frac{1}{q}}. \]
Equality occurs if and only if \((\forall i, a_i=0)\) or \((\forall i, b_i=0)\) or \((\exists c_1,c_2\in\R_*^+\forall i,c_1a_i^p=c_2b_i^q)\).
\end{theorem}
\begin{proof}
If \(\forall i, a_i=0\) or \(\forall i, b_i=0\), both sides of the inequality are zero, and equality occurs trivially.
So we consider the case where the right hand side of the inequality is greater than 0.
Then it is equivalent to prove
\[ \frac{\sum_{i=1}^n a_ib_i}{\left(\sum_{i=1}^n a_i^p\right)^{\frac{1}{p}}\left(\sum_{i=1}^n b_i^q\right)^{\frac{1}{q}}}\leq1. \]
Then, we define
\[ x_j\coloneqq\frac{a_j}{\left(\sum_{i=1}^n a_i^p\right)^{\frac{1}{p}}}, \quad y_j\coloneqq\frac{b_j}{\left(\sum_{i=1}^n b_i^q\right)^{\frac{1}{q}}}, \quad \forall j\in[1,n]\cap\N. \]
Then it is equivalent to prove
\[ \sum_{j=1}^n x_jy_j\leq1. \]
Apply Young's inequality, we have
\[ x_jy_j\leq\frac{x_j^p}{p}+\frac{y_j^q}{q}, \quad \forall j\in[1,n]\cap\N. \]
So their sum satisfies
\[ \sum_{j=1}^n x_jy_j\leq \sum_{j=1}^n\left(\frac{x_j^p}{p}+\frac{y_j^q}{q}\right)=\frac{1}{p}\sum_{j=1}^n x_j^p+\frac{1}{q}\sum_{j=1}^n y_j^q. \]
According to the definition,
\[ \sum_{j=1}^n x_j^p=\frac{\sum_{j=1}^n a_j^p}{\sum_{i=1}^n a_i^p}=1. \]
Similarly,
\[ \sum_{j=1}^n y_j^q=\frac{\sum_{j=1}^n b_j^q}{\sum_{i=1}^n b_i^q}=1. \]
Therefore,
\[ \frac{\sum_{i=1}^n a_ib_i}{\left(\sum_{i=1}^n a_i^p\right)^{\frac{1}{p}}\left(\sum_{i=1}^n b_i^q\right)^{\frac{1}{q}}}=\sum_{j=1}^n x_jy_j\leq\frac{1}{p}+\frac{1}{q}=1. \]
\end{proof}



\begin{theorem}[Minkowski's inequality]\label{minkowski}
Let \(p\in(1,+\infty)\) be strictly positive real number.
Two sequences of positive real numbers \(a_i,b_i\in\R^+,\forall i\in[1,n]\cap\N\) satisfies
\[ \left[\sum_{i=1}^n (a_i+b_i)^p\right]^\frac{1}{p}\leq\left(\sum_{i=1}^n a_i^p\right)^\frac{1}{p}+\left(\sum_{i=1}^n b_i^p\right)^\frac{1}{p}. \]
\end{theorem}
\begin{proof}
Define
\[ q\coloneqq\frac{p}{p-1} \]
such that \(\frac{1}{p}+\frac{1}{q}=1\) is satisfied.
It follows from the H\"{o}lder's inequality that
\begin{align*}
\sum_{i=1}^n(a_i+b_i)^p&=\sum_{i=1}^na_i(a_i+b_i)^{p-1}+\sum_{i=1}^nb_i(a_i+b_i)^{p-1}\\
&\leq \left(\sum_{i=1}^na_i^p\right)^\frac{1}{p}\left[\sum_{i=1}^n(a_i+b_i)^{p}\right]^\frac{1}{q}
+\left(\sum_{i=1}^nb_i^p\right)^\frac{1}{p}\left[\sum_{i=1}^n(a_i+b_i)^{p}\right]^\frac{1}{q}\\
&=\left[\left(\sum_{i=1}^na_i^p\right)^\frac{1}{p}+\left(\sum_{i=1}^nb_i^p\right)^\frac{1}{p}\right]\left[\sum_{i=1}^n(a_i+b_i)^{p}\right]^\frac{1}{q}
\end{align*}
If \(\sum_{i=1}^n (a_i+b_i)^p=0\), the equality occurs trivially.
So we consider the case where \(\sum_{i=1}^n (a_i+b_i)^p>0\), where we can divide both sides by \(\left[\sum_{i=1}^n (a_i+b_i)^p\right]^{1/q}\) without changing the direction of the inequality.
Notice that \(1-\frac{1}{q}=\frac{1}{p}\), we get
\[ \left[\sum_{i=1}^n(a_i+b_i)^p\right]^\frac{1}{p}\leq\left(\sum_{i=1}^n a_i^p\right)^\frac{1}{p}+\left(\sum_{i=1}^n b_i^p\right)^\frac{1}{p}. \]
\end{proof}


With Minkowski's inequality, it is obvious that equation (\ref{matrixnorm}) defines a family of matrix norm, since the triangle inequality required by the norm is a special case of Minkowski's inequality.



\begin{enumerate}
	\item Show that
	\begin{equation}
	\sum_{m=-\infty}^{\infty} y[m]y[m-i]=\sum_{i=1}^p a_i\sum_{m=-\infty}^{\infty} y[m-i]y[m-i], \quad i=1,\cdots,p.
	\end{equation}
	Since those sums are infinite they cannot be computed, and as such need to be truncated.\cite{karris}
	This can be achieved by applying the covariance method, which consists in windowing the error
	\begin{equation}
	E_n=\sum_{m=n}^{n+N-1}\left(y[m]-\sum_{i=1}^p a_i y[m-k] \right)^2.
	\end{equation}
	\item Prove that
	\begin{align*}
	\phi_n(k,0)&=\sum_{i=0}^p a_i\phi_n(k,i), \\
	\phi_n(k,i)&=\sum_{m=n}^{n+N-1} y[m-k]y[m-k].
	\end{align*}
	\item Conclude that
	\begin{equation}
	\begin{pmatrix} \phi(1,0)\\ \vdots \\ \phi(p,0) \end{pmatrix}=\begin{pmatrix} \phi(1,1) & \cdots & \phi(1,p) \\ \vdots & & \vdots \\ \phi(p,1) & \cdots & \phi(p,p) \end{pmatrix}\begin{pmatrix} a_1 \\ \vdots \\ a_p \end{pmatrix}.
	\end{equation}
	Determining the optimal value for the \(a_i, 1\leq i\leq p\), implies inverting the matrix $\Phi$.
	This can be achieved through Cholesky decomposition.\cite{gtm216,gtm135}
\end{enumerate}