%!TEX root = main.tex
\titleformat{\section}{\normalfont\large\bfseries}{\thesection}{1em}{}
\renewcommand{\chaptername}{Project}
\renewcommand{\thesection}{\arabic{section}}


\chapter{Linear Prediction of Speech}
\begin{center}
Guangting Yu, University of Michigan - Shanghai Jiao Tong University Joint Institute, Minhang, Shanghai, 200135, China
\end{center}


\section*{Abstract}
This paper talks about numerical methods.


\section{Background}
Speech production is the result of an excitation signal generated by the contraction of the lungs when they expel air.\cite{dutoit}
It is then modified by resonances when passing through the trachea, the vocal cords, the mouth cavity, as well as various muscles.\cite{tam59}
The excitation signal is either created by the opening and closing of the vocal cords, or by a continuous flow of air.\cite{gtm181}
Introduced in the early 1960s by Fant, \textit{the source-filter} model assumes that the glottis and vocal tract are fully uncoupled.\cite{corless}
This initial idea was reused to develop the \textit{Linear Predictive} (LP) model for speech production.\cite{tam39}

In this model the speech signal is the output \(y[n]\) of an \textit{all-pole filter}\footnote{A filter whose frequency response function goes infinite at specific frequencies} \(1/A(z)\) excited by \(x[n]\).\cite{golan}
Calling \(Y(z)\) and \(X(z)\) the Z-transform of the speech and excitation, respectively, the model is described by\cite{utm}
\begin{equation}
Y(z)=\frac{X(z)}{1-\sum_{i=0}^p a_i z^{-i}}=\frac{X(z)}{A_p}.
\end{equation}

Applying the inverse Z-transform to this equation we observe that the speech can be linearly predicted from the previous $p$ samples and some excitation:
\begin{equation}
y[n] = x[n]+\sum_{i=1}^p a_i y[n-i].
\end{equation}


Our goal is to explain as much as possible of \(y[n]\) through the $a_i$ , \textit{i.e.}, we look at \(x[n]\) as an error, and we strive at rendering it as small and simple as possible.\cite{gtm135}
For the sake of clarity we therefore rename \(x[n]\) into \(e[n]\).\cite{cc12}
The question we want to answer is how to select the $a_i$ such as to minimize the energy
\begin{equation}
E=\sum_{m=-\infty}^\infty e^2[m].
\end{equation}


\begin{enumerate}
	\item Show that
	\begin{equation}
	\sum_{m=-\infty}^{\infty} y[m]y[m-i]=\sum_{i=1}^p a_i\sum_{m=-\infty}^{\infty} y[m-i]y[m-i], \quad i=1,\cdots,p.
	\end{equation}
	Since those sums are infinite they cannot be computed, and as such need to be truncated.\cite{karris}
	This can be achieved by applying the covariance method, which consists in windowing the error
	\begin{equation}
	E_n=\sum_{m=n}^{n+N-1}\left(y[m]-\sum_{i=1}^p a_i y[m-k] \right)^2.
	\end{equation}
	\item Prove that
	\begin{align*}
	\phi_n(k,0)&=\sum_{i=0}^p a_i\phi_n(k,i), \\
	\phi_n(k,i)&=\sum_{m=n}^{n+N-1} y[m-k]y[m-k].
	\end{align*}
	\item Conclude that
	\begin{equation}
	\begin{pmatrix} \phi(1,0)\\ \vdots \\ \phi(p,0) \end{pmatrix}=\begin{pmatrix} \phi(1,1) & \cdots & \phi(1,p) \\ \vdots & & \vdots \\ \phi(p,1) & \cdots & \phi(p,p) \end{pmatrix}\begin{pmatrix} a_1 \\ \vdots \\ a_p \end{pmatrix}.
	\end{equation}
	Determining the optimal value for the \(a_i, 1\leq i\leq p\), implies inverting the matrix $\Phi$.
	This can be achieved through Cholesky decomposition.\cite{gtm216,gtm135}
\end{enumerate}


\section{Linear algebra}



\begin{definition}[Hermitian matrix]
A Matrix \(A\in\C^{n\times n}\) is \emph{Hermitian} if it is equal to the conjugate transpose of itself, \textit{i.e.},
\begin{equation*}
A=\bar{A}^T.
\end{equation*}
\end{definition}


\begin{definition}[Eigenvalues ad eigenvectors]
A number \(\lambda\in\C\) is called an \emph{eigenvalue} of \(A\in\C^{n\times n}\) if there exists \(v\in\C^n\) such that \(v\neq0\) and
\begin{equation*}
Av=\lambda v.
\end{equation*}
Such a vector $v$ is called an \emph{eigenvector}.
\end{definition}


\begin{definition}[Spectrum]
The \emph{spectrum} of \(A\in\C^{n\times n}\), denoted as \(\sigma_A\), is the set of all the eigenvalues of $A$, \textit{i.e.},
\begin{equation*}
\sigma_A\coloneqq\Set{\lambda}{\lambda \text{ is a eigenvalue of }A}
\end{equation*}
\end{definition}


\begin{definition}[Spectrum radius]
The \emph{spectrum radius} of \(A\in\C^{n\times n}\), denoted as \(\rho(A)\), is the largest absolute value of all its eigenvalues, \textit{i.e.},
\begin{equation*}
\rho(A)\coloneqq\max_{\lambda\in\sigma_A}|\lambda|.
\end{equation*}
\end{definition}


\begin{theorem}
If a matrix \(A\in\C^{n\times n}\) is Hermitian, then \(\sigma_A\subset\R\).
\end{theorem}
\begin{proof}
Take any eigenvalue \(\lambda\in\sigma_A\) and its corresponding eigenvector \(v\in\C^n\).
By definition,
\begin{equation}
Av=\lambda v.
\end{equation}
Multiply both sides by the conjugate transpose of $v$,
\begin{equation}\label{hermitianproof}
\bar{v}^T Av=\lambda \bar{v}^T v.
\end{equation}
The left hand side of the equation has the conjugate transpose
\begin{equation}
\overline{\bar{v}^T Av}^T=\bar{v}^T\bar{A}^T v.
\end{equation}
Since $A$ is Hermitian, we plug in \(\bar{A}^T=A\) on the right hand side and
\begin{equation}
\overline{\bar{v}^T Av}^T=\bar{v}^TA v,
\end{equation}
which implies that \(\bar{v}^T Av\) is also Hermitian.
Furthermore, \(\bar{v}^T Av\in\C^{1\times1}\) implies that it is a real number.
Then, according to equation \ref{hermitianproof}
\begin{equation}
\lambda=\frac{\bar{v}^T Av}{\bar{v}^T v}.
\end{equation}
The denominator is the inner product of $v$ with itself defined on \(\C^n\), so \(\bar{v}^T v\in\R_*^+\) since \(v\neq0\).
Both the numerator and the denominator are real, so the quotient $\lambda$ is also real.
Since $\lambda$ is arbitrarily selected from \(\sigma_A\), all elements in \(\sigma_A\) are real.
Furthermore, \(\sigma_A\) has at most $n$ elements, so it is a proper subset of 
\end{proof}

\begin{theorem}
If \(A,B\in\C^{n\times n}\), then \(\sigma_{AB}=\sigma_{BA}\).
\end{theorem}
\begin{proof}
Select \(\lambda\in\sigma_{AB}\).
\item If \(\lambda=0\), then there exists \(v\in\C^n,v\neq0\) such that \(ABv=\lambda v=0\).
So the matrix $AB$ is not full rank, and thus \(\det(AB)=0\).
Then \(\det(BA)=\det(AB)=0\), which implies $BA$ is not full rank either.
Thus, there exists a nonzero vector \(u\in\C^n\) such that \(BAu=0\).
This means 0 is also an eigenvalue of $BA$.
\item If \(\lambda\neq0\), then there exists \(v\in\C^n,v\neq0\) such that \(ABv=\lambda v\).
Then, multiply both sides by $B$,
\begin{equation}
BABv=\lambda Bv
\end{equation}
This also means \(BA\) has $\lambda$ as its eigenvalue with corresponding eigenvector \(Bv\).
Combine both cases, \(\lambda\in\sigma_{BA}\).
Since $\lambda$ is arbitrarily selected from \(\sigma_{AB}\), we have \(\sigma_{AB}\subseteq\sigma_{BA}\).
Exchange the name of $A$ and $B$, we also have \(\sigma_{AB}\supseteq\sigma_{BA}\).
Therefore, \(\sigma_{AB}=\sigma_{BA}\).
This directly implies \(\rho(AB)=\rho(BA)\).
\end{proof}

\begin{definition}[Norm]
Given a vector space $V$ over a subfield \(\mathbb{F}\subseteq\C\), a \emph{norm} on $V$ is a function \(||\cdot||:V\to\R\) with the following properties:
\item For all \(c\in\mathbb{F}\) and all \(u,v\in V\),
\begin{enumerate}
	\item \(||v||\geq0\);
	\item \(||cv||=|c|\cdot||v||\);
	\item \(||u+v||\leq||u||+||v||\);
	\item \(||v||=0\implies v=0\).
\end{enumerate}
\end{definition}

\begin{theorem}
The set of functions on \(A\in\C^{n\times n}\) with \(a_{ij}\in A,1\leq i,j\leq n\),
\begin{equation}
||A||_p\coloneqq
\begin{cases} 
\left(\sum_{1\leq i,j\leq n}|a_{ij}|^p\right)^{\frac{1}{p}} & p\in\N^*\\
\max_{1\leq i,j\leq n}|a_{ij}| & p=+\infty
\end{cases}
\end{equation}
define a set of matrix norms.
\end{theorem}
