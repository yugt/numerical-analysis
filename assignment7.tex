%!TEX root = main.tex
\titleformat{\section}{\normalfont\large\bfseries}{Exercise \thesection\ --- }{1pt}{}
% \setcounter{chapter}{6}
\renewcommand{\chaptername}{Assignment}
\chapter{Iteration Methods}


\section{Lipschitz Continuity vs. Differentiability}
A function \(f:\R\to\R\) is differentiable in some open interval \(\mathcal{I}\subset\R\) if it is differentiable at every point of \(\mathcal{I}\), and it is Lipschitz continuous if there is a constant \(c\geq0\) such that
\[ |f(x_1)-f(x_2)\leq c|x_1-x_2| \qquad \forall x_1,x_2\in\mathcal{I}. \]
\begin{enumerate}
	\item Give a function, with proof, that is differentiable but its derivative is not bounded in some open interval.
	\item Suppose $f$ is differentiable and $f'$ is bounded in some open interval.
	Prove that $f$ is Lipschitz continuous in that interval.
	\item Give a function, with proof, that is differentiable but not Lipschitz continuous in some open interval.
	\item Give a function, with proof, that is Lipschitz continuous but not differentiable in some open interval.
\end{enumerate}



\section{Fixed Point Iteration Convergence Condition}
A fixed point of a function \(g(x)\) is a real number \(x^∗\) such that
\[ g(x^∗ )=x^∗. \]
Assume the followings:
\begin{enumerate}
	\item The function $g$ and its derivative $g'$ are continuous on \([a, b]\), \textit{i.e.}, \(g,g'\in\mathcal{C}[a,b\).
	\item The function $g$ is bounded below by $a$ and above by $b$, \textit{i.e.}, \(g(x)\in[a, b] \forall x\in[a,b]\).
	\item The initial point \(x_0\) is an interior point of \([a,b]\), \textit{i.e.}, \(x_0\in(a, b)\).
\end{enumerate}
Show the followings are true.
\begin{enumerate}
	\item If \(0\leq|g'(x)|<1, \forall x\in[a,b]\), the fixed-point iteration will converge to the unique fixed point \(x^∗\in[a,b]\).
	\item If \(|g'(x)|>1 \forall x\in[a,b]\), then the fixed-point iteration will never converge to \(x^∗\).
\end{enumerate}


\section{Root Finding}
Choose a suitable numerical method to find the smallest positive root and the second smallest positive root of the equation
\[ \tan x = 4x \]
correct to 3 decimal places.
Explain your choice.


\section{Order of Convergence}
Suppose that the sequence {a n } converges to the number L, and there is a constant
0 <λ< ∞
such that
|a n+1 − L|
= λ
n→∞ |a n − L|
lim
then the sequence is said to converge linearly to L. If
|a n+1 − L|
= λ
n→∞ |a − L| 2
n
lim
then the sequence is said to converge quadratically to L. If
|a n+1 − L|
= λ
|a n − L| α
lim
n→∞
then the sequence is said to converge to L with order of convergence α. The constant λ is
called the asymptotic error. A positive sequence {E n } is said to has an order of at least α
and a rate of at most λ if there is a sequence {a n } that has an order α and a rate of λ such
that
E n ≤ a n
for all n.
(a) (2 points) Find the order of convergence and the rate of convergence of the sequence
a n =
1
n
for
n ≥ 1
(b) (2 points) Examine the order of convergence and the rate of convergence of {b n } where
b 2k =
1
ln k
and
b 2k+1 =
1
k
for
k ≥ 1
(c) (2 points) Use the precise definition above to show the method of fixed-point iteration
leads an error sequence that has at least linear convergence.
(d) (2 points) Find the asymptotic error constant for the method of fixed-point iteration
when it has at least quadratic convergence.
(e) (2 points) Show the order of convergence of the secant method is
√
1+ 5
2


\section{Order of Roots}
Suppose that f (x) and its derivatives f 0 (x), . . . , f (m) (x) are defined and continuous on an
interval containing x ∗ , we say that f (x) = 0 has a root of order m at x = x ∗ if and only if
f (x ∗ ) = 0,
f 0 (x ∗ ) = 0,
f 00 (x ∗ ) = 0,
···
f (m−1) (x ∗ ) = 0,
and f (m) (x ∗ ) 6 = 0
The positive integer m is known as the multiplicity of the root. A root of order m = 1 is
often called a simple root, and if m > 1, it is called a multiple root


(a) (2 points) Prove by induction that there exists a continuous function h(x) so that f (x)
can be expressed as the product
f (x) = (x − x ∗ ) m h(x),
where
h(x ∗ ) 6 = 0
if the equation f (x) has a root of order m at x = x ∗ .
(b) (2 points) Show the following modified Newton’s iteration will produce a sequence that
converges quadratically to x ∗
x k = x k−1 − m
f (x k−1 )
f 0 (x k−1 )
if the original Newton’s method produces a sequence converges linearly to the root
x = x ∗ of order m > 1.
(c) (2 points) In practice, it is unlikely that m is known, in those cases the following
version of modified Newton’s method can be applied to accelerate convergence
x k = x k−1 −
u(x k−1 )
u 0 (x k−1 )
where
u(x) =
f (x)
f 0 (x)
Explain how this helps to speed up convergence.
(d) (2 points) Zero is a root of multiplicity of 3 for the function
f (x) = sin(x 3 ).
Start with x 0 = 1 and compute x 1 , x 2 , . . ., x 6 first by using the iteration formula in
part (b) and then by using the iteration formula in part (c). What can you conclude
based on the values that you have computed.
(e) (2 points) Compare Newton’s method and the modified version of it in part (c) by
finding all the roots of the following polynomial correct to 6 decimal places.
f (x) = x 5 − 11x 4 + 46x 3 − 90x 2 + 81x − 27


\section{Problem with Root Finding Methods}
Provide an example, if exists, of root-finding problems that satisfy the following criteria:
(a) (2 points) It can be solved by Bisection but not by using fixed-point iteration.
(b) (2 points) It can be solved by using fixed-point iteration but not by Newton’s method.


\section{Complex Root Finding for Analytic Functions}
Let f : C → C be analytic. The following iteration formula can be used to solve f (z) = 0.
f (z k )
z k+1 = z k − "
f 0 (z k ) −
!
#
f 00 (z k )
f (z k )
2f 0 (z k )
where f 0 (z) and f 00 (z) are the usual complex derivatives of f (z).
(a) (1 point (bonus)) For the following function
f (z) = z 7 − 1
depict the relationship between the initial point z 0 = x 0 + iy 0 and the number of
iteration required for the sequence to be considered to have converged, that is,
|z k − z k−1 | 2 < 
where  = 0.0001.
(b) (1 point (bonus)) What happens to the relationship in part (a) if we use the criterion


 2
2 
where  = 0.0001.
